\documentclass[12pt, a4paper]{article}

\usepackage[
  margin=1in
]{geometry}

\usepackage[T1, T2A]{fontenc}
\usepackage[utf8]{inputenc}
\usepackage[russian]{babel}
\usepackage{textcomp}
\usepackage{minted}
\usepackage{hyperref}

\begin{document}

\begin{titlepage}
  \centering
  \textsc{Новосибирский государственный технический университет}\par
  \vspace{1mm}
  Кафедра прикладной математики\par
  \vspace{4cm}
  \textsc{Практическая работа \textnumero 2}\par
  {\huge\bfseries Элементарные методы шифрования\par}
  \vspace{1cm}
  {\scriptsize ФПМИ, ПМ-24\par}
  \vspace{1mm}
  {\itshape\large Параскун И., Шакиров П., Герасименко В.\par}
  \vfill
  {\small преподаватель\par}
  \vspace{1mm}
  \textsc{Ступаков Илья Михайлович}
  \vfill
  \large{Новосибирск, 2024}
\end{titlepage}

\newpage

\section{Цель}
Знакомство с элементарными методами шифрования данных и криптоанализа.

\section{Шифрование методом простой замены}
\subsection{Разработать программу для генерации ключа}

\begin{minted}{go}
type PrivateKey map[rune]rune

func GenerateKey(rand io.Reader) (*PrivateKey, error) {
	var rnd [33]byte

	_, err := rand.Read(rnd[:])

	if err != nil {
		return nil, err
	}

	var key PrivateKey = make(map[rune]rune, 34)

	for i := 1072; i < 1105; i++ {
		key[rune(i)] = rune(i)
	}

	for i := 32; i > 0; i-- {
		rnd[i] = uint8(float64(rnd[i]) / 0xff.p0 * float64(i))

		a := rune(i) + 1072
		b := rune(rnd[i]) + 1072

		if key[b] == 1104 {
			key[b] = 1105
		}

		key[a], key[b] = key[b], key[a]
	}

	key[1105] = key[1104]
	delete(key, 1104)

	return &key, nil
}

func (key *PrivateKey) String() string {
	var sb strings.Builder

	for i := 1072; i < 1078; i++ {
		sb.WriteRune((*key)[rune(i)])
	}

	sb.WriteRune((*key)[1105])

	for i := 1078; i < 1104; i++ {
		sb.WriteRune((*key)[rune(i)])
	}

	return sb.String()
}
\end{minted}

\begin{minted}{console}
[...]$ crypto rot kgen -h

Generate private key using Fisher-Yates shuffle.

Usage:
  crypto rot kgen [flags]

Flags:
  -h, --help         help for kgen
  -o, --out string   output file, default: stdout

[...]$ crypto rot kgen

ихъарцюфйщэнпляужмебгёктсьчшводыз
\end{minted}

\subsection{Разработать программу для шифрования текста}

\begin{minted}{go}
type Encoder struct {
	key *PrivateKey
	src io.RuneReader
}

func NewEncoder(key *PrivateKey, src io.RuneReader) *Encoder {
	return &Encoder{key, src}
}

func (enc *Encoder) Read(b []byte) (s int, err error) {
	for {
		r, _, err := enc.ReadRune()

		if err != nil {
			return s, err
		}

		if utf8.RuneLen(r) > len(b) {
			return s, nil
		}

		l := utf8.EncodeRune(b, r)

		b = b[l:]
		s = s + l
	}
}

func (enc *Encoder) ReadRune() (rune, int, error) {
	for {
		r, _, err := enc.src.ReadRune()

		if err != nil {
			return utf8.RuneError, 0, err
		}

		r = unicode.ToLower(r)
		r, ok := (*enc.key)[r]

		if !ok {
			continue
		}

		return r, utf8.RuneLen(r), nil
	}
}
\end{minted}

\begin{minted}{console}
[...]$ crypto rot -h
Simple substitution encryption.

Usage:
  crypto rot [flags]
  crypto rot [command]

Available Commands:
  kgen        Generate private key using Fisher-Yates shuffle.

Flags:
  -d, --dec          decrypt given input
  -h, --help         help for rot
  -i, --in string    input file, default: stdin
  -k, --key string   key file (required)
  -o, --out string   output file, default: stdout

Use "crypto rot [command] --help" for more information about a command.

[...]$ ls -lh
... 1.9M Oct 26 20:29 essay.txt
...   67 Oct 26 20:28 ru.key

[...]$                                                    \
          date +%S.%3N;                                   \
          crypto rot -i essay.txt -o essay.rot -k ru.key; \
          date +%S.%3N

21.703
21.770

[...]$ ls -lh
... 1.7M Oct 26 20:34 essay.rot
... 1.9M Oct 26 20:29 essay.txt
...   67 Oct 26 20:28 ru.key
\end{minted}

\subsection{Зашифровать эссе}

\begin{minted}{text}
Криптография — это наука о защите информации, играющая ключевую 
роль в современном мире. Она обеспечивает безопасность данных, 
позволяя передавать личные и финансовые сведения без риска их перехвата. 
Современные методы обеспечивают конфиденциальность и целостность 
информации, что особенно важно в эпоху цифровизации и глобализации.
\end{minted}

\begin{minted}{console}
[...]$ crypto rot -i essay.in ru.key
хтжяцйчтлгждшцйнлёхлйулвжцфжнгйтълпжжжчтлэвлдхоэефиёэтйозищйит
фъфннйъъжтфйнлймфщяфежилфцмфуйялщнйщцзюлннкряйуийоддяфтфюлилцз
оженкфжгжнлнщйикфщифюфнждмфутжщхлжряфтфрилцлщйитфъфннкфъфцйюкй
мфщяфежилэцхйнгжюфнпжлознйщцзжпфойщцнйщцзжнгйтълпжжецййщймфннй
илынйишяйрёпжгтйижулпжжжчоймложулпжж
\end{minted}

\textit{\href{https://istupakov.ddns.net:4006/chat/message/b8af3253-be42-47d9-a560-489afcfa8bdd}{Click me!}}


\section{Ручная дешифровка}

\begin{minted}{text}
рцъчдхрныщаъеегпабдафъъафщевцбртещерзыяельеэшащъьегекбеьзрцвцщ
умзъэдбцэогнощамньещэдёдаещхермцэубзсщатъщеьфъъакцищьелмевбеъщ
уцьщерняаракцщуъьеиьдкабдаднбдтцэубзыьлюэшкбцьапдьцвбзмцъяатще
мшьэшащъшъяеъегбеъщуцрюнмабщдреьцщуъьеощеётнлрабдшдрцлмзчэшщуб
цкьеяреъцмдлцщрцюдьцопдмдёдщцщаэшллнщцъбмьбтньэгвцбскшлэжцйцит
юдёшръымтябьощёщлтюьоиеашъмчшкёчфйидщй
\end{minted}

\begin{minted}{text}
> зьтсйнмъврчлхюэицшыпкабежудяфёщог (начальный ключ)

ипжйщлиесэфжццятфхщфыжжфыэцзпхивцэциасъцкбцнрфэжбцяцухцбаипзпэ
шёажнщхпнюяеюэфёебцэнщьщфцэлциёпншхагэфвжэцбыжжфупоэбцкёцзхцжэ
шпбэциеъфифупэшжбцобщуфхщфщехщвпншхасбкмнрухпбфтщбпзхаёпжъфвэц
ёрбнрфэжржъцжцяхцжэшпимеёфхэщицбпэшжбцюэцьвекифхщрщипкёайнрэшх
пубцъицжпёщкпэипмщбпютщёщьщэпэфнрккеэпжхёбхвебнязпхгуркнчпдпов
мщьрижсёвъхбюэьэквмбюоцфржёйруьйыдощэд
\end{minted}

\begin{minted}{text}
> цгтсйамувдылкюбезръщнхэижчпяфёшоь (учтена данная информация)

расшифруйтесообщениеыссеытозанрвоторпйъокяохэетсяоболнояпразат
жёпсхинахюбуютеёуяотхиьиеотфорёахжнпгтевстояысселацтяокёозност
жаяторуъерелатжсяоцяилениеиунивахжнпйякмхэлнаяещияазнпёасъевто
ёэяхэетсэсъособностжармуёентирояатжсяоютоьвукрениэиракёпшхэтжн
аляоъросаёикатрамияающиёиьитатехэккутаснёянвуяхбзанглэкхчадацв
миьэрсйёвъняютьтквмяюцоеэсёшэльшыдцитд
\end{minted}

\begin{minted}{text}
> цгьюкаивлдытэмбеяръщнхсжёчпйзуфош

расшифруйтесообщениеэссеэтожанркоторыйпозволяетсвободновыражат
ьмыслиналюбуютемувотличиеотформальныхтекстовэсседаётвозможност
ьавторупередатьсвоёвидениеиуникальныйвзгляднавещиважнымаспекто
мявляетсяспособностьаргументироватьсвоюточкузренияиразмышлятьн
адвопросамизатрагивающимичитателяззутаснмвнкувлбжанхдязлцаъаёк
гичярсймкпнвютчтзкгвюёоеясмшядчшэъёитъ
\end{minted}

\textit{\textbf{цгьюкаивлдытэмбеяръщнхсжёчпйзуфош}}


\end{document}

