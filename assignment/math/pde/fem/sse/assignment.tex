\documentclass[12pt, a4paper]{article}

\usepackage[
  margin=1in
]{geometry}

\usepackage[T1, T2A]{fontenc}
\usepackage[utf8]{inputenc}
\usepackage[russian]{babel}
\usepackage{minted}
\usepackage{hyperref}
\usepackage{tabularray}
\usepackage{mathtools, nccmath}
\usepackage{float}
\usepackage{tikz}
\usepackage{circuitikz}

\begin{document}

\begin{titlepage}
  \centering
  \textsc{Новосибирский государственный технический университет}\par
  \vspace{1mm}
  Кафедра прикладной математики\par
  \vspace{4cm}
  \textsc{Курсовой проект}\par
  {\huge\bfseries Метод конечных элементов для эллиптических задач\par}
  \vspace{1cm}
  {\scriptsize ФПМИ, ПМ-24\par}
  \vspace{1mm}
  {\itshape\large Параскун И.\par}
  \vfill
  {\small преподаватели\par}
  \vspace{2mm}
  \textsc{Рояк М. Э., д.т.н., профессор}\par
  \vspace{1mm}
  \textsc{Сивенкова А. П., ассистент}\par
  \vfill
  \large{Новосибирск, 2024}
\end{titlepage}

\newpage

\setcounter{page}{2}
\tableofcontents

\newpage

\section{Постановка задачи}
Построить МКЭ для уравнения эллиптического типа

\begin{align}
  \label{1} -\nabla(\lambda\nabla u) + \gamma u = f, \ f = \frac{\partial{g}}{\partial{x}}
\end{align}

\noindent в декартовой системе координат

\begin{align}
  -\frac{\partial}{\partial{x}}(\lambda \frac{\partial{u}}{\partial{x}})
    -\frac{\partial}{\partial{y}}(\lambda \frac{\partial{u}}{\partial{y}})
    -\frac{\partial}{\partial{z}}(\lambda \frac{\partial{u}}{\partial{z}})
    + \gamma u = f
\end{align}

\vspace{5mm}
\noindent с учётом следующих условий:

\begin{itemize}
\item Краевые условия всех типов:

\begin{fleqn}[20pt]
\begin{align}
  &u|_{S_1} = u_g \\[1ex]
  &\lambda \frac{\partial{u}}{\partial{n}}\bigg\rvert_{S_2} = \theta \\[1ex]
  &\lambda \frac{\partial{u}}{\partial{n}}\bigg\rvert_{S_3} + 
    \beta (u\rvert_{S_3} - u{\beta}) = 0
\end{align}
\end{fleqn}

\item Базисные функции - линейные на прямоугольных параллелепипедах.
\item Коэффициент диффузии - кусочно-постоянная функция.
\item Матрицы в разреженном строчном формате.
\item Решение СЛАУ с использованием ЛОС с ILU факторизацией.
\end{itemize}

\section{Теоретическая часть}
\subsection{Вариационная постановка}

Эквивалентная вариационная постановка в форме уравнения Галёркина
для уравнения \ref{1}:

\begin{align}
  \begin{split}\label{2}
     \int_{\Omega}\lambda\nabla{u}\cdot\nabla{v_0}d\Omega 
  + &\int_{\Omega}\gamma u v_0 d\Omega
  +  \int_{S_3}\beta u v_0 dS = \\
  = &\int_{\Omega}fv_0d\Omega + 
       \int_{S_2}\theta v_0dS + 
       \int_{S_3}\beta u_{\beta}v_0dS,
      \quad \forall v_0 \in H_0^1
  \end{split}
\end{align}

\noindent Аппроксимация уравнения \ref{2} на конечномерных подпространствах $V_g^h$ и $V_0^h$ 
получается заменой функций $u \in H_g^1$ и $v_0 \in H_0^1$ 
на функции $u^h \in V_g^h$ и $v_0^h \in V_0^h$ соответственно:

\begin{align}
  \begin{split}\label{3}
       \int_{\Omega}\lambda\nabla{u^h}\cdot\nabla{v_0^h}d\Omega 
    + &\int_{\Omega}\gamma u^h v_0^h d\Omega
    +  \int_{S_3}\beta u^h v_0^h dS = \\
    = &\int_{\Omega}fv_0^hd\Omega + 
       \int_{S_2}\theta v_0^hdS + 
       \int_{S_3}\beta u_{\beta}v_0^hdS,
      \quad \forall v_0^h \in V_0^h
  \end{split}
\end{align}

\subsection{Конечноэлементная СЛАУ}

\noindent Раскладывая функции $u^h$ и $v_0^h$ по базису, переходим к конечноэлементной СЛАУ

\begin{align}
  \begin{split}
    \sum_{j=1}^{n}&(
      \int_{\Omega}\lambda\nabla{\psi_j}\cdot\nabla{\psi_i}d\Omega 
    + \int_{\Omega}\gamma \psi_j \psi_i d\Omega
    + \int_{S_3}\beta \psi_j \psi_i dS
  ) q_j = \\
    &= \int_{\Omega}f \psi_i d\Omega 
    + \int_{S_2}\theta \psi_i dS 
    + \int_{S_3}\beta u_{\beta} \psi_i dS,
  \quad i \in N_0
  \end{split} \\[1ex]
  \sum_{j}&(q_j \psi_j)\bigg\rvert_{S_1} = u_g
\end{align}

\vspace{3mm}
\noindent или в матричном виде

\begin{align}
    \textbf{Aq = b}
\end{align}

\vspace{3mm}
\noindent где компоненты матрицы $\textbf{A}$ и вектора $\textbf{b}$ определяются соотношениями

\begin{align}
  &A_{ij} = 
  \begin{cases} 
      \int_{\Omega}\lambda\nabla{\psi_j}\cdot\nabla{\psi_i}d\Omega 
        + \int_{\Omega}\gamma \psi_j \psi_i d\Omega
        + \int_{S_3}\beta \psi_j \psi_i dS & i \in N_0 \\
      \delta_{ij} & i \notin N_0 \\
  \end{cases} \\[1ex]
  &b_i =
  \begin{cases} 
      \int_{\Omega}f \psi_i d\Omega 
        + \int_{S_2}\theta \psi_i dS 
        + \int_{S_3}\beta u_{\beta} \psi_i dS & i \in N_0 \\
      u_g(\textbf{x}_i) & i \notin N_0 \\
  \end{cases}
\end{align}

\vspace{3mm}
\noindent Преобразование в сумму интегралов по конечным элементам даёт более удобную форму записи:

\begin{align}
  &G_{ij} = \int_{\Omega}\lambda\nabla{\psi_i}\cdot\nabla{\psi_j}d\Omega =
            \sum_{k}\int_{\Omega_k}\lambda\nabla{\psi_i}\cdot\nabla{\psi_j}d\Omega \\
  &M_{ij} = \int_{\Omega}\gamma \psi_i \psi_j d\Omega =
            \sum_{k}\int_{\Omega_k}\gamma \psi_i \psi_j d\Omega \\
  &M_{ij}^{S_3} = \int_{S_3}\beta \psi_i \psi_j dS =
                  \sum_{l}\int_{S_3^l}\beta \psi_i \psi_j dS \\
  &b_{i} = \int_{\Omega}f\psi_{i}d\Omega = 
           \sum_{k}\int_{\Omega_k}f\psi_{i}d\Omega \\
  \label{4} &b_{i}^{S_2} = \int_{S_2}\theta\psi_{i}dS = 
                            \sum_{p}\int_{S_2^p}\theta\psi_{i}dS \\
  \label{5} &b_{i}^{S_3} = \int_{S_3}\beta u_{\beta} \psi_{i}dS = 
                            \sum_{l}\int_{S_3^l}\beta u_{\beta}\psi_{i}dS
\end{align}

\subsection{Трилинейные базисные функции}

Трилинейные базисные функции на прямоугольном параллелепипеде $\Omega_{psr}$ строятся следующим образом:

\begin{align}
  \label{6} X_1(x) &= \frac{x_{p+1} - x}{h_x}, & X_2(x) &= \frac{x-x_p}{h_x}, & h_x &= x_{p+1}-x_p, \\
  \label{7} Y_1(y) &= \frac{y_{s+1} - y}{h_y}, & Y_2(y) &= \frac{y-y_s}{h_y}, & h_y &= y_{s+1}-y_s, \\
  \label{8} Z_1(z) &= \frac{z_{r+1} - z}{h_z}, & Z_2(z) &= \frac{z-z_r}{h_z}, & h_z &= z_{r+1}-z_r.
\end{align}

\noindent Локальные базисные функции представляются в виде произведения функций \ref{6}, \ref{7} и \ref{8}:

\begin{align}
  \label{9} \hat{\psi_i} = X_{\mu(i)}Y_{\nu(i)}Z_{\vartheta(i)}
\end{align}

\noindent гдe

\begin{align}
  &\mu(i) = ((i-1)\ mod\ 2) + 1 \\
  &\nu(i) = \left(\left[ \frac{i-1}{2} \right] mod\ 2\right) + 1 \\
  &\vartheta(i) = \left[ \frac{i-1}{4} \right] + 1
\end{align}

\subsection{Элементы локальных матриц}

\noindent Выражение для вычисления компонент локальных матриц на прямоугольном параллелепипеде $\Omega_{psr}$:

\begin{align}
  &\hat{G}_{ij} = \int_{x_p}^{x_{p+1}}
                  \int_{y_s}^{y_{s+1}}
                  \int_{z_r}^{z_{r+1}}
                    \lambda(
                      \frac{\partial{\hat{\psi_i}}}{\partial{x}}
                      \frac{\partial{\hat{\psi_j}}}{\partial{x}} +
                      \frac{\partial{\hat{\psi_i}}}{\partial{y}}
                      \frac{\partial{\hat{\psi_j}}}{\partial{y}} +
                      \frac{\partial{\hat{\psi_i}}}{\partial{z}}
                      \frac{\partial{\hat{\psi_j}}}{\partial{z}}
                    )dxdydz \\[1ex]
  &\hat{M}_{ij} = \int_{x_p}^{x_{p+1}}
                  \int_{y_s}^{y_{s+1}}
                  \int_{z_r}^{z_{r+1}}
                  \gamma\hat{\psi}_i\hat{\psi}_j dxdydz \\[1ex]
  &\hat{M}_{ij}^{S_3} = \int_{\xi_1}^{\xi_2}
                        \int_{\zeta_1}^{\zeta_2}
                        \beta\psi_{i}\psi_{j} d\xi d\zeta
\end{align}

\vspace{5mm}
\noindent Учитывая вид базисных функций \ref{9}:

\begin{align*}
  \hat{\textbf{G}}^x &= \frac{1}{h_x}\textbf{G}^c & \hat{\textbf{M}}^x &= h_x\textbf{M}^c \\
  \hat{\textbf{G}}^y &= \frac{1}{h_y}\textbf{G}^c & \hat{\textbf{M}}^y &= h_y\textbf{M}^c \\
  \hat{\textbf{G}}^z &= \frac{1}{h_z}\textbf{G}^c & \hat{\textbf{M}}^z &= h_z\textbf{M}^c
\end{align*}

\noindent где

\begin{align}
  \textbf{G}^c &= 
    \begin{pmatrix}
       1 & -1 \\
      -1 &  1
    \end{pmatrix} &
  \textbf{M}^c &= \frac{1}{6}
    \begin{pmatrix}
      2 & 1 \\
      1 & 2
    \end{pmatrix}
\end{align}

\vspace{5mm}
\noindent Тогда

\begin{align}
  \begin{split}
    \hat{G}_{ij} = \hat{\lambda}(
      &\hat{G}_{\mu(i)\mu(j)}^{x} \hat{M}_{\nu(i)\nu(j)}^{y} \hat{M}_{\vartheta(i)\vartheta(j)}^{z} + \\
      &\hat{M}_{\mu(i)\mu(j)}^{x} \hat{G}_{\nu(i)\nu(j)}^{y} \hat{M}_{\vartheta(i)\vartheta(j)}^{z} + \\
      &\hat{M}_{\mu(i)\mu(j)}^{x} \hat{M}_{\nu(i)\nu(j)}^{y} \hat{G}_{\vartheta(i)\vartheta(j)}^{z}
    )
  \end{split} \\[1ex]
  \hat{M}_{ij} = \hat{\gamma}(
    &\hat{M}_{\mu(i)\mu(j)}^x 
     \hat{M}_{\nu(i)\nu(j)}^y 
     \hat{M}_{\vartheta(i)\vartheta(j)}^z
  ) \\[1ex]
  \hat{M}_{ij}^{S_3} = \hat{\beta}(
    &\hat{M}_{\mu(i)\mu(j)}^{\xi} 
     \hat{M}_{\nu(i)\nu(j)}^{\zeta} 
  )
\end{align}

\subsection{Элементы локальных векторов}

\noindent Выражение для вычисления компонент локальных векторов на 
прямоугольном параллелепипеде $\Omega_{psr}$:

\begin{align}
  \begin{split}
    \hat{b}_i 
      &= \int_{x_p}^{x_{p+1}}
        \int_{y_s}^{y_{s+1}}
        \int_{z_r}^{z_{r+1}}
          f\hat{\psi}_i dxdydz \\
      &= \sum_{j}\hat{q}_j^g
        \int_{x_p}^{x_{p+1}}
        \int_{y_s}^{y_{s+1}}
        \int_{z_r}^{z_{r+1}}
          \frac{\partial{\hat{\psi}_j}}{\partial{x}}\hat{\psi}_i dxdydz
  \end{split}
\end{align}

\vspace{3mm}
\noindent Учитывая вид базисных функций \ref{9}:

\begin{align}
  \begin{split}
    \hat{b}_i 
    = \sum_{j}\hat{q}_j^g
      \int_{x_p}^{x_{p+1}}&X_{\mu(i)}\frac{\partial{X_{\mu(j)}}}{\partial{x}}dx
      \int_{y_s}^{y_{s+1}}Y_{\nu(i)}Y_{\nu(j)}dy
      \int_{z_r}^{z_{r+1}}Z_{\vartheta(i)}Z_{\vartheta(j)}dz = \\
    = &\sum_{j}\hat{q}_j^g 
      X_{\mu(i)\mu(j)} 
      \hat{M}_{\nu(i)\nu(j)}^y 
      \hat{M}_{\vartheta(i)\vartheta(j)}^z
  \end{split}
\end{align}

\noindent где

\begin{align}
  \textbf{X} = 
    \begin{pmatrix}
      -\frac{1}{2} & \frac{1}{2} \\
      -\frac{1}{2} & \frac{1}{2}
    \end{pmatrix}
\end{align}

\vspace{3mm}
\noindent Для правой части, заданной в явном виде:

\begin{align}
  \begin{split}
    \hat{b}_i 
    = \sum_{j}\hat{f}_j
      \int_{x_p}^{x_{p+1}}&X_{\mu(i)}Y_{\mu(j)}dx
      \int_{y_s}^{y_{s+1}}Y_{\nu(i)}Y_{\nu(j)}dy
      \int_{z_r}^{z_{r+1}}Z_{\vartheta(i)}Z_{\vartheta(j)}dz = \\
    = &\sum_{j}\hat{f}_j 
      \hat{M}_{\mu(i)\mu(j)}^x 
      \hat{M}_{\nu(i)\nu(j)}^y 
      \hat{M}_{\vartheta(i)\vartheta(j)}^z
  \end{split}
\end{align}

\vspace{3mm}
\noindent Вклад краевых условий \ref{4} и \ref{5}:

\begin{align}
  &\hat{b}_{i}^{S_2} =  \int_{\xi_1}^{\xi_2}
                        \int_{\zeta_1}^{\zeta_2}
                        \hat{\theta}\psi_{i} d\xi d\zeta \\
  &\hat{b}_{i}^{S_3} =  \int_{\xi_1}^{\xi_2}
                      \int_{\zeta_1}^{\zeta_2}
                      \hat{\beta}\hat{u}_{\beta}\psi_{i} d\xi d\zeta
\end{align}

\begin{align}
  &\hat{g}^{ip} 
    = \sum_{i=1}^{8}
        g(\hat{x}_i, \hat{y}_i, \hat{z}_i)\hat{\psi}_i(x, y, z)
    = \sum_{i=1}^{8}
        \hat{q}_{i}^{g}\hat{\psi}_i(x, y, z) \\
  &\hat{\theta}^{ip} 
    = \sum_{i=1}^{4}
        \hat{\theta}(\hat{x}_i, \hat{y}_i, \hat{z}_i)\hat{\psi}_i(x, y, z)
    = \sum_{i=1}^{4}
        \hat{\theta}_{i}\hat{\psi}_i(x, y, z) \\
  &\hat{u}_{\beta}^{ip} 
    = \sum_{i=1}^{4}
        \hat{u}_{\beta}(\hat{x}_i, \hat{y}_i, \hat{z}_i)\hat{\psi}_i(x, y, z)
    = \sum_{i=1}^{4}
        \hat{u}_{\beta i}\hat{\psi}_i(x, y, z)
\end{align}

\begin{align}
  \begin{split}
    \hat{b}_i^{S_2} 
      = \hat{\theta}_1 \hat{M}_{\mu(i)\mu(1)}^{\xi} \hat{M}_{\nu(i)\nu(1)}^{\chi} + 
        \hat{\theta}_2 \hat{M}_{\mu(i)\mu(2)}^{\xi} \hat{M}_{\nu(i)\nu(2)}^{\chi} + \\
      + \hat{\theta}_3 \hat{M}_{\mu(i)\mu(3)}^{\xi} \hat{M}_{\nu(i)\nu(3)}^{\chi} +
        \hat{\theta}_4 \hat{M}_{\mu(i)\mu(4)}^{\xi} \hat{M}_{\nu(i)\nu(4)}^{\chi}
  \end{split} \\[1ex]
  \begin{split}
    \hat{b}_i^{S_3} 
      = \hat{\beta}(
        \hat{u}_{\beta 1} \hat{M}_{\mu(i)\mu(1)}^{\xi} \hat{M}_{\nu(i)\nu(1)}^{\chi} + 
        \hat{u}_{\beta 2} \hat{M}_{\mu(i)\mu(2)}^{\xi} \hat{M}_{\nu(i)\nu(2)}^{\chi} + \\
      + \hat{u}_{\beta 3} \hat{M}_{\mu(i)\mu(3)}^{\xi} \hat{M}_{\nu(i)\nu(3)}^{\chi} +
        \hat{u}_{\beta 4} \hat{M}_{\mu(i)\mu(4)}^{\xi} \hat{M}_{\nu(i)\nu(4)}^{\chi}
      )
  \end{split}
\end{align}

\section{Описание разработанных программ}
\subsection{Узел}

\inputminted[firstline=4, lastline=20]{c}{/home/mehandes/c/src/github.com/meha4j/math/pde/fem/sse/include/fem/sse/vtx.h}

\subsection{Конечный элемент}

\inputminted[firstline=9, lastline=45]{c}{/home/mehandes/c/src/github.com/meha4j/math/pde/fem/sse/include/fem/sse/hex.h}

\subsection{Грань}

\inputminted[firstline=8, lastline=77]{c}{/home/mehandes/c/src/github.com/meha4j/math/pde/fem/sse/include/fem/sse/fce.h}

\subsection{Структура алгоритма}

\inputminted[firstline=9, lastline=53]{c}{/home/mehandes/c/src/github.com/meha4j/math/pde/fem/sse/include/fem/sse/fem.h}

\section{Описание тестирования программ}
\subsection{Тестирование на одном конечном элементе}
\subsubsection{Описание задачи}

Простейшая расчетная область для тестирования трилинейных элементов выглядит следующим образом:

\vspace{5mm}
\begin{figure}[H]
\centering
\resizebox{1\textwidth}{!}{%
\begin{circuitikz}
\tikzstyle{every node}=[font=\large]

\node at (7.5,8) [circ] {};
\node at (16.25,8) [circ] {};
\node at (7.5,9.25) [circ] {};
\node at (16.25,9.25) [circ] {};
\node at (21.25,10.5) [circ] {};
\node at (21.25,11.75) [circ] {};
\node at (12.5,10.5) [circ] {};
\node at (12.5,11.75) [circ] {};
\draw [line width=2pt, short] (7.5,8) -- (16.25,8);
\draw [line width=2pt, short] (7.5,9.25) -- (7.5,8);
\draw [line width=2pt, short] (16.25,9.25) -- (16.25,8);
\draw [line width=2pt, short] (7.5,9.25) -- (16.25,9.25);
\draw [line width=2pt, short] (16.25,8) -- (21.25,10.5);
\draw [line width=2pt, short] (16.25,9.25) -- (21.25,11.75);
\draw [line width=2pt, short] (21.25,11.75) -- (21.25,10.5);
\draw [line width=2pt, short] (7.5,9.25) -- (12.5,11.75);
\draw [line width=2pt, short] (12.5,11.75) -- (21.25,11.75);
\draw [line width=2pt, dashed] (7.5,8) -- (12.5,10.5);
\draw [line width=2pt, dashed] (12.5,11.75) -- (12.5,10.5);
\draw [line width=2pt, dashed] (12.5,10.5) -- (21.25,10.5);
\draw [line width=0.6pt, dashed] (7.5,8) -- (5,6.75);
\draw [line width=0.6pt, dashed] (16.25,8) -- (13.75,6.75);
\draw [line width=0.6pt, dashed] (16.25,8) -- (18,8);
\draw [line width=0.6pt, dashed] (21.25,10.5) -- (23,10.5);
\draw [line width=0.6pt, dashed] (7.5,9.25) -- (6.25,9.25);
\node [font=\normalsize] at (7.5,8) {\textbf{1}};
\node [font=\normalsize] at (7.5,8) {\textbf{1}};
\draw [line width=0.5pt, ->, >=Stealth] (1.5,5) -- (4,6.25);
\draw [line width=0.5pt, ->, >=Stealth] (1.5,5) -- (1.5,8.25);
\node [font=\large] at (7.25,8.25) {\textbf{1}};
\node [font=\large] at (16,8.25) {\textbf{2}};
\node [font=\large] at (12.25,10.75) {\textbf{3}};
\node [font=\large] at (21,10.75) {\textbf{4}};
\node [font=\large] at (7.25,9.5) {\textbf{5}};
\node [font=\large] at (16,9.5) {\textbf{6}};
\node [font=\large] at (12.25,12) {\textbf{7}};
\node [font=\large] at (21,12) {\textbf{8}};
\node [font=\large] at (5,7.25) {10};
\node [font=\large] at (13.75,7.25) {20};
\node [font=\large] at (18.25,7.75) {0};
\node [font=\large] at (23.25,10.25) {8};
\node [font=\large] at (6,9.5) {1};
\node [font=\large] at (1,8.25) {z};
\node [font=\large] at (3.75,6.75) {y};
\draw [line width=0.5pt, ->, >=Stealth] (1.5,5) -- (4.75,5);
\node [font=\large] at (4.75,4.5) {x};
\end{circuitikz}
}%
\label{fig:simple}
\end{figure}

\noindent где

\begin{align*}
  S_2^1 &= \left[ 1, 3, 5, 7 \right] \\
  S_2^2 &= \left[ {3, 4, 7, 8} \right] \\
  S_2^3 &= \left[ {5, 6, 7, 8} \right] \\
  S_3^1 &= \left[ {2, 4, 6, 8} \right] \\
  S_3^2 &= \left[ {1, 2, 5, 6} \right] \\
  S_3^3 &= \left[ {1, 2, 3, 4} \right]
\end{align*}

\noindent Значения коэффициентов $\lambda$ и $\gamma$ выбраны в виде:

\begin{align}
  \lambda &= 5 & \gamma &= 0.4
\end{align}

\noindent Решение - полином вида

\begin{align}
  u = 5 + 0.2x + y + 30z + 0.5xy + xz + 10yz + xyz
\end{align}

\noindent Исходя из этого

\begin{align}
  &f = 0.4(5 + 0.2x + y + 30z + 0.5xy + xz + 10yz + xyz) \\[1ex]
  &g = 2x + 0.04x^2 + 0.4xy + 12xz + 0.1x^2y + 0.2x^2z + 4xyz + 0.2x^2yz \\[1ex]
  &\theta|_{S_2^1} = -1 - 2.5y - 5z - 5yz \\[1ex]
  &\theta|_{S_2^2} = 5 + 2.5x + 50z + 5xz \\[1ex]
  &\theta|_{S_2^3} = 150 + 5x + 50y + 5xy
\end{align}

\noindent При условии, что $\beta = 10$:

\begin{align}
  &u_{\beta}|_{S_3^1} = 9.1 + 11.25y + 50.5z + 30.5yz \\[1ex]
  &u_{\beta}|_{S_3^2} = 4.5 - 0.05x + 25z + 0.5xz \\[1ex]
  &u_{\beta}|_{S_3^3} = -10 - 0.3x - 4y
\end{align}

\subsubsection{Подпрограмма тестирования}

\inputminted[firstline=38, lastline=122]{cpp}{/home/mehandes/c/src/github.com/meha4j/math/pde/fem/sse/test/SolidStateEquationSingleElementTest.cc}

\subsubsection{Результат}

\begin{minted}{console}
7.0000000e+00
9.0000000e+00
5.5000000e+01
9.7000000e+01
4.7000000e+01
5.9000000e+01
2.5500000e+02
3.8700000e+02
\end{minted}

\subsection{Тестирование на сетке с разрывом коэффициентов}
\subsubsection{Описание задачи}

\vspace{5mm}
\begin{figure}[H]
\centering
\resizebox{1\textwidth}{!}{%
\begin{circuitikz}
\tikzstyle{every node}=[font=\normalsize]
\node at (7.5,8.5) [circ] {};
\node at (10,9.75) [circ] {};
\node at (11.25,8.5) [circ] {};
\node at (13.75,9.75) [circ] {};
\node at (17.5,8.5) [circ] {};
\node at (20,9.75) [circ] {};
\node at (7.5,11) [circ] {};
\node at (11.25,11) [circ] {};
\node at (17.5,11) [circ] {};
\node at (20,12.25) [circ] {};
\node at (10,12.25) [circ] {};
\node at (7.5,16) [circ] {};
\node at (11.25,16) [circ] {};
\draw [ line width=1.5pt](7.5,16) to[short] (7.5,8.5);
\draw [line width=1.5pt, short] (7.5,8.5) -- (17.5,8.5);
\draw [line width=1.5pt, short] (17.5,11) -- (17.5,8.5);
\draw [line width=1.5pt, short] (20,12.25) -- (20,9.75);
\draw [line width=1.5pt, short] (11.25,11) -- (11.25,8.5);
\draw [line width=1.5pt, short] (11.25,16) -- (11.25,11);
\draw [line width=1.5pt, short] (7.5,16) -- (11.25,16);
\draw [line width=1.5pt, dashed] (11.25,8.5) -- (13.75,9.75);
\draw [line width=1.5pt, dashed] (10,9.75) -- (7.5,8.5);
\draw [line width=1.5pt, dashed] (10,9.75) -- (20,9.75);
\node at (13.75,17.25) [circ] {};
\node at (10,17.25) [circ] {};
\node at (13.75,12.25) [circ] {};
\draw [line width=1.5pt, short] (7.5,16) -- (10,17.25);
\draw [line width=1.5pt, short] (11.25,16) -- (13.75,17.25);
\draw [line width=1.5pt, dashed] (13.75,12.25) -- (13.75,9.75);
\draw [line width=1.5pt, dashed] (10,12.25) -- (13.75,12.25);
\draw [line width=1.5pt, dashed] (10,12.25) -- (7.5,11);
\draw [line width=1.5pt, dashed] (10,17.25) -- (10,9.75);
\draw [line width=1.5pt, short] (10,17.25) -- (13.75,17.25);
\draw [line width=1.5pt, short] (7.5,11) -- (17.5,11);
\draw [line width=1.5pt, short] (20,12.25) -- (17.5,11);
\draw [line width=1.5pt, short] (17.5,8.5) -- (20,9.75);
\draw [line width=0.5pt, ->, >=Stealth] (2.5,6) -- (5,6);
\draw [line width=0.5pt, ->, >=Stealth] (2.5,6) -- (2.5,8.5);
\draw [line width=0.5pt, ->, >=Stealth] (2.5,6) -- (4.5,7);
\node [font=\normalsize] at (2.25,8.5) {z};
\node [font=\normalsize] at (5,5.75) {x};
\node [font=\normalsize] at (4.25,7.25) {y};
\node [font=\normalsize] at (5,7) {};
\draw [line width=1pt, dashed] (10,9.75) -- (12,9.75);
\draw [line width=0.3pt, dashed] (7.5,16) -- (6.25,16);
\draw [line width=0.3pt, dashed] (7.5,11) -- (6.25,11);
\draw [line width=0.3pt, dashed] (7.5,8.5) -- (6.5,8);
\draw [line width=0.3pt, dashed] (17.5,8.5) -- (18.75,8.5);
\draw [line width=0.3pt, dashed] (20,9.75) -- (21.25,9.75);
\draw [line width=0.3pt, dashed] (10,9.75) -- (5,7.25);
\draw [line width=0.3pt, dashed] (11.25,8.5) -- (8.75,7.25);
\draw [line width=0.3pt, dashed] (17.5,8.5) -- (15,7.25);
\draw [line width=1.5pt, short] (13.75,17.25) -- (13.75,12.25);
\draw [line width=1.5pt, short] (13.75,12.25) -- (20,12.25);
\draw [line width=1.5pt, short] (11.25,11) -- (13.75,12.25);
\node [font=\normalsize] at (11,8.75) {\textbf{2}};
\node [font=\normalsize] at (7.25,8.75) {\textbf{1}};
\node [font=\normalsize] at (17.25,8.75) {\textbf{3}};
\node [font=\normalsize] at (9.75,10) {\textbf{4}};
\node [font=\normalsize] at (13.5,10) {\textbf{5}};
\node [font=\normalsize] at (19.75,10) {\textbf{6}};
\node [font=\normalsize] at (7.25,11.25) {\textbf{7}};
\node [font=\normalsize] at (11,11.25) {\textbf{8}};
\node [font=\normalsize] at (17.25,11.25) {\textbf{9}};
\node [font=\normalsize] at (9.5,12.5) {\textbf{10}};
\node [font=\normalsize] at (13.25,12.5) {\textbf{11}};
\node [font=\normalsize] at (19.75,12.5) {\textbf{12}};
\node [font=\normalsize] at (7.25,16.25) {\textbf{13}};
\node [font=\normalsize] at (11,16.25) {\textbf{14}};
\node [font=\normalsize] at (9.75,17.5) {\textbf{15}};
\node [font=\normalsize] at (13.5,17.5) {\textbf{16}};
\node [font=\normalsize] at (5.25,7.75) {0};
\node [font=\normalsize] at (9,7.75) {1};
\node [font=\normalsize] at (15.25,7.75) {5};
\node [font=\normalsize] at (19,8.25) {0};
\node [font=\normalsize] at (21.5,9.5) {2.5};
\node [font=\normalsize] at (6.25,11.25) {2};
\node [font=\normalsize] at (6.25,16.25) {7};
\node [font=\normalsize] at (8.75,10) {$\Omega_1$};
\node [font=\normalsize] at (15.5,10.25) {$\Omega_2$};
\node [font=\normalsize] at (8.75,13.75) {$\Omega_3$};
\end{circuitikz}
}%

\label{fig:cff}
\end{figure}

\noindent Значения коэффициентов $\lambda$ и $\gamma$ выбраны в виде:

\begin{align*}
  &\lambda = 
  \begin{cases} 
      1, & \Omega^1 \\
      2, & \Omega^2 \\
      \frac{2}{3}, & \Omega^3
  \end{cases}
  &\gamma = 
  \begin{cases} 
      4, & \Omega^1 \\
      0, & \Omega^2 \\
      1, & \Omega^3
  \end{cases}
\end{align*}

\noindent Решение - полином вида

\vspace{1mm}
\begin{align*}
  &u = 
  \begin{cases} 
      2x(yz + z + y + 1), & \Omega^1 \\
      1 + x + y + z + xy + xz + yz + xyz, & \Omega^2 \\
      3xz(y + 1), & \Omega^3
  \end{cases}
\end{align*}
\vspace{1mm}

\noindent При условии, что $\beta = 2$: 

\begin{align*}
  S_1 &= \left[ 1, 4, 7, 10, 13, 15 \right] & u|_{S_1} &= 0 \\
  S_2^1 &= \left[ 1, 2, 7, 8 \right] & \theta|_{S_2^1} &= -2x(z + 1) \\
  S_2^2 &= \left[ 7, 8, 13, 14 \right] & \theta|_{S_2^2} &= -2xz \\
  S_2^3 &= \left[ 2, 3, 8, 9 \right] & \theta|_{S_2^3} &= -2(1 + x + z + xz) \\
  S_2^4 &= \left[ 3, 6, 9, 12 \right] & \theta|_{S_2^4} &= 2(1 + y + z + yz) \\
  S_2^5 &= \left[ 8, 9, 11, 12 \right] & \theta|_{S_2^5} &= 2(1 + x + y + xy) \\
  S_2^6 &= \left[ 8, 11, 14, 16 \right] & \theta|_{S_2^6} &= 2z(1 + y) \\
  S_2^7 &= \left[ 13, 14, 15, 16 \right] & \theta|_{S_2^7} &= 2x(1 + y) \\
  S_2^8 &= \left[ 1, 2, 4, 5 \right] & \theta|_{S_2^8} &= -2x(1 + y) \\
  S_2^9 &= \left[ 2, 3, 5, 6 \right] & \theta|_{S_2^9} &= -2(1 + x + y + xy) \\
  S_3^1 &= \left[ 4, 5, 10, 11 \right] & u_{\beta}|_{S_3^1} &= 8x(z + 1) \\
  S_3^2 &= \left[ 10, 11, 15, 16 \right] & u_{\beta}|_{S_3^2} &= 11.5xz \\
  S_3^3 &= \left[ 5, 6, 11, 12 \right] & u_{\beta}|_{S_3^3} &= 4.5(1 + x + z + xz)
\end{align*}

\vspace{1mm}

\begin{align*}
  &f = 
  \begin{cases} 
      8x(yz + z + y + 1), & \Omega^1 \\
      0, & \Omega^2 \\
      3xz(y + 1), & \Omega^3
  \end{cases}
\end{align*}
\vspace{1mm}

\subsection{Тестирование на сетке с внутренним узлом}
\subsubsection{Описание задачи}

\vspace{5mm}
\begin{figure}[H]
\centering
\resizebox{1\textwidth}{!}{%
\begin{circuitikz}
\tikzstyle{every node}=[font=\normalsize]
\node at (7.5,8.5) [circ] {};
\node at (10,9.75) [circ] {};
\node at (11.25,8.5) [circ] {};
\node at (13.75,9.75) [circ] {};
\node at (17.5,8.5) [circ] {};
\node at (20,9.75) [circ] {};
\node at (7.5,11) [circ] {};
\node at (11.25,11) [circ] {};
\node at (17.5,11) [circ] {};
\node at (20,12.25) [circ] {};
\node at (10,12.25) [circ] {};
\node at (7.5,16) [circ] {};
\node at (11.25,16) [circ] {};
\draw [ line width=1.5pt](7.5,16) to[short] (7.5,8.5);
\draw [line width=1.5pt, short] (7.5,8.5) -- (17.5,8.5);
\draw [line width=1.5pt, short] (17.5,11) -- (17.5,8.5);
\draw [line width=1.5pt, short] (20,12.25) -- (20,9.75);
\draw [line width=1.5pt, short] (11.25,11) -- (11.25,8.5);
\draw [line width=1.5pt, short] (11.25,16) -- (11.25,11);
\draw [line width=1.5pt, short] (7.5,16) -- (11.25,16);
\draw [line width=1.5pt, dashed] (11.25,8.5) -- (13.75,9.75);
\draw [line width=1.5pt, dashed] (10,9.75) -- (7.5,8.5);
\draw [line width=1.5pt, dashed] (10,9.75) -- (20,9.75);
\node at (13.75,17.25) [circ] {};
\node at (10,17.25) [circ] {};
\node at (13.75,12.25) [circ] {};
\node at (11.25,13.5) [circ] {};
\node at (7.5,13.5) [circ] {};
\draw [line width=1.5pt, short] (7.5,16) -- (10,17.25);
\draw [line width=1.5pt, short] (11.25,16) -- (13.75,17.25);
\draw [line width=1.5pt, dashed] (13.75,12.25) -- (13.75,9.75);
\draw [line width=1.5pt, dashed] (10,12.25) -- (13.75,12.25);
\draw [line width=1.5pt, dashed] (10,12.25) -- (7.5,11);
\draw [line width=1.5pt, dashed] (11.25,11) -- (13.75,12.25);
\draw [line width=1.5pt, dashed] (13.75,12.25) -- (20,12.25);
\draw [line width=1.5pt, dashed] (10,17.25) -- (10,9.75);
\draw [line width=1.5pt, short] (7.5,13.5) -- (17.5,13.5);
\draw [line width=1.5pt, short] (10,17.25) -- (13.75,17.25);
\draw [line width=1.5pt, dashed] (7.5,13.5) -- (10,14.75);
\draw [line width=1.5pt, dashed] (10,14.75) -- (13.75,14.75);
\draw [line width=1.5pt, short] (7.5,11) -- (17.5,11);
\draw [line width=1.5pt, short] (20,12.25) -- (17.5,11);
\draw [line width=1.5pt, short] (17.5,8.5) -- (20,9.75);
\draw [line width=1.5pt, short] (17.5,13.5) -- (20,14.75);
\draw [line width=1.5pt, short] (20,14.75) -- (20,12.25);
\draw [line width=1.5pt, short] (17.5,13.5) -- (17.5,11);
\node at (13.75,14.75) [circ] {};
\node at (17.5,13.5) [circ] {};
\node at (20,14.75) [circ] {};
\node at (10,14.75) [circ] {};
\draw [line width=1.5pt, short] (13.75,17.25) -- (13.75,14.75);
\draw [line width=1.5pt, short] (11.25,13.5) -- (13.75,14.75);
\draw [line width=1.5pt, short] (13.75,14.75) -- (20,14.75);
\draw [line width=1.5pt, dashed] (13.75,14.75) -- (13.75,12.25);
\node at (19,11.75) [circ] {};
\node at (9,11.75) [circ] {};
\node at (9,9.25) [circ] {};
\node at (19,9.25) [circ] {};
\node at (9,14.25) [circ] {};
\node at (12.75,14.25) [circ] {};
\node at (19,14.25) [circ] {};
\node at (9,16.75) [circ] {};
\node at (12.75,16.75) [circ] {};
\node at (12.75,9.25) [circ] {};
\draw [line width=1.5pt, short] (12.75,16.75) -- (12.75,14.25);
\draw [line width=1.5pt, short] (12.75,14.25) -- (19,14.25);
\draw [line width=1.5pt, short] (19,14.25) -- (19,9.25);
\draw [line width=1.5pt, short] (9,16.75) -- (12.75,16.75);
\draw [line width=1.5pt, dashed] (9,16.75) -- (9,9.5);
\draw [line width=1.5pt, dashed] (9,9.25) -- (19,9.25);
\draw [line width=1.5pt, dashed] (12.75,14.25) -- (12.75,9.25);
\draw [line width=1.5pt, dashed] (9,11.75) -- (19,11.75);
\draw [line width=1.5pt, dashed] (9,14.25) -- (12.75,14.25);
\draw [line width=0.5pt, ->, >=Stealth] (2.5,6) -- (5,6);
\draw [line width=0.5pt, ->, >=Stealth] (2.5,6) -- (2.5,8.5);
\draw [line width=0.5pt, ->, >=Stealth] (2.5,6) -- (4.5,7);
\node [font=\normalsize] at (2.25,8.5) {z};
\node [font=\normalsize] at (5,5.75) {x};
\node [font=\normalsize] at (4.25,7.25) {y};
\node [font=\normalsize] at (7.25,8.75) {\textbf{1}};
\node [font=\normalsize] at (5,7) {};
\node [font=\normalsize] at (11,8.75) {\textbf{2}};
\node [font=\normalsize] at (17.25,8.75) {\textbf{3}};
\node [font=\normalsize] at (8.75,9.5) {\textbf{4}};
\node [font=\normalsize] at (12.5,9.5) {\textbf{5}};
\node [font=\normalsize] at (18.75,9.5) {\textbf{6}};
\node [font=\normalsize] at (9.75,10) {\textbf{7}};
\node [font=\normalsize] at (13.5,10) {\textbf{8}};
\node [font=\normalsize] at (19.75,10) {\textbf{9}};
\node [font=\normalsize] at (7.25,11.25) {\textbf{10}};
\node [font=\normalsize] at (11,11.25) {\textbf{11}};
\node [font=\normalsize] at (17.25,11.25) {\textbf{12}};
\node [font=\normalsize] at (8.75,12) {\textbf{13}};
\node [font=\normalsize] at (12.5,12) {\textbf{14}};
\node [font=\normalsize] at (18.75,12) {\textbf{15}};
\node [font=\normalsize] at (9.75,12.5) {\textbf{16}};
\node [font=\normalsize] at (13.5,12.5) {\textbf{17}};
\node [font=\normalsize] at (19.75,12.5) {\textbf{18}};
\node [font=\normalsize] at (7.25,13.75) {\textbf{19}};
\node [font=\normalsize] at (11,13.75) {\textbf{20}};
\node [font=\normalsize] at (17.25,13.75) {\textbf{21}};
\node [font=\normalsize] at (8.75,14.5) {\textbf{22}};
\node [font=\normalsize] at (12.5,14.5) {\textbf{23}};
\node [font=\normalsize] at (18.75,14.5) {\textbf{24}};
\node [font=\normalsize] at (9.75,15) {\textbf{25}};
\node [font=\normalsize] at (13.5,15) {\textbf{26}};
\node [font=\normalsize] at (19.75,15) {\textbf{27}};
\node [font=\normalsize] at (7.25,16.25) {\textbf{28}};
\node [font=\normalsize] at (11,16.25) {\textbf{29}};
\node [font=\normalsize] at (8.75,17) {\textbf{30}};
\node [font=\normalsize] at (12.5,17) {\textbf{31}};
\node [font=\normalsize] at (9.75,17.5) {\textbf{32}};
\node [font=\normalsize] at (13.5,17.5) {\textbf{33}};
\draw [line width=1pt, dashed] (10,9.75) -- (12,9.75);
\draw [line width=0.3pt, dashed] (7.5,16) -- (6.25,16);
\draw [line width=0.3pt, dashed] (7.5,13.5) -- (6.25,13.5);
\draw [line width=0.3pt, dashed] (7.5,11) -- (6.25,11);
\draw [line width=0.3pt, dashed] (7.5,8.5) -- (6.5,8);
\draw [line width=0.3pt, dashed] (17.5,8.5) -- (18.75,8.5);
\draw [line width=0.3pt, dashed] (20,9.75) -- (21.25,9.75);
\draw [line width=0.3pt, dashed] (19,9.25) -- (20.25,9.25);
\draw [line width=0.3pt, dashed] (10,9.75) -- (5,7.25);
\draw [line width=0.3pt, dashed] (11.25,8.5) -- (8.75,7.25);
\draw [line width=0.3pt, dashed] (17.5,8.5) -- (15,7.25);
\node [font=\normalsize] at (5.25,7.75) {0};
\node [font=\normalsize] at (9,7.75) {3};
\node [font=\normalsize] at (15.25,7.75) {8};
\node [font=\normalsize] at (19,8) {0};
\node [font=\normalsize] at (20.5,8.75) {1.5};
\node [font=\normalsize] at (21.75,9.5) {2.5};
\node [font=\normalsize] at (6.25,11.25) {2};
\node [font=\normalsize] at (6.25,13.75) {4};
\node [font=\normalsize] at (6.25,16.25) {6};
\end{circuitikz}
}%

\label{fig:inner}
\end{figure}

\noindent где

\begin{align*}
  S_1^1 &= \left[ 1, 2, 28, 29 \right] \\
  S_1^2 &= \left[ 2, 3, 20, 21 \right] \\
  S_1^3 &= \left[ 7, 8, 32, 33 \right] \\
  S_1^4 &= \left[ 1, 3, 7, 9 \right] \\
  S_1^4 &= \left[ 8, 9, 26, 27 \right] \\
  S_2^1 &= \left[ 1, 7, 28, 32 \right] \\
  S_2^2 &= \left[ {3, 9, 21, 27} \right] \\
  S_3^1 &= \left[ {28, 29, 32, 33} \right] \\
  S_3^2 &= \left[ {20, 26, 29, 33} \right] \\
  S_3^3 &= \left[ {20, 21, 26, 27} \right] \\
\end{align*}

\noindent Значения коэффициентов $\lambda$ и $\gamma$ выбраны в виде:

\begin{align}
  \lambda &= 5 & \gamma &= 0.4
\end{align}

\noindent Решение - полином вида

\begin{align}
  u = 5 + 0.2x + y + 30z + 0.5xy + xz + 10yz + xyz
\end{align}

\noindent Исходя из этого

\begin{align}
  &f = 0.4(5 + 0.2x + y + 30z + 0.5xy + xz + 10yz + xyz) \\[1ex]
  &g = 2x + 0.04x^2 + 0.4xy + 12xz + 0.1x^2y + 0.2x^2z + 4xyz + 0.2x^2yz \\[1ex]
  &u|_{S_1} = 0 \\[1ex]
  &\theta|_{S_2^1} = -1 - 2.5y - 5z - 5yz \\[1ex]
  &\theta|_{S_2^2} = 1 + 2.5y + 5z + 5yz
\end{align}

\noindent При условии, что $\beta = 10$:

\begin{align}
  &u_{\beta}|_{S_3^1} = 7xy + 6.7x + 66y + 200 \\[1ex]
  &u_{\beta}|_{S_3^2} = 13.5yz + 2.75y + 33.5z + 5.7 \\[1ex]
  &u_{\beta}|_{S_3^3} = 5xy + 4.7x + 46y + 140
\end{align}


\section{Проведенные исследования и выводы}
\section{Тексты основных модулей программ}

\end{document}

