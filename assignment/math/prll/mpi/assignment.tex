\documentclass[12pt, a4paper]{article}

\usepackage[
  margin=1in
]{geometry}

\usepackage[T1, T2A]{fontenc}
\usepackage[utf8]{inputenc}
\usepackage[russian]{babel}
\usepackage{minted}
\usepackage{tabularray}
\usepackage{siunitx}

\begin{document}

\begin{titlepage}
  \centering
  \textsc{Новосибирский государственный технический университет}\par
  \vspace{1mm}
  Кафедра прикладной математики\par
  \vspace{4cm}
  \textsc{Практическая работа \textnumero 3}\par
  {\huge\bfseries Основы работы в MPI\par}
  \vspace{1cm}
  {\scriptsize ФПМИ, ПМ-24\par}
  \vspace{1mm}
  {\itshape\large Параскун И., Шакиров П., Герасименко В.\par}
  \vfill
  {\small преподаватель\par}
  \vspace{1mm}
  \textsc{Домников Петр Александрович}
  \vfill
  \large{Новосибирск, 2024}
\end{titlepage}

\newpage
\setcounter{page}{2}

\section{Простейшее MPI приложение}
\subsection{Постановка задачи}
\begin{itemize}
  \item Написать программу с использованием MPI, в которой каждый выводит свой номер и их общее число.
  \item Создать командный файл, который запускает на выполнение написанную MPI программу в количестве 10 экземпляров.
\end{itemize}

\subsection{Текст программы}

\begin{minted}{c}
#define COM MPI_COMM_WORLD

int main(int argc, char* argv[argc]) {
  if ((errno = MPI_Init(&argc, &argv)))
    goto err;

  int pc;
  int id;

  if ((errno = MPI_Comm_size(COM, &pc)))
    goto err;

  if ((errno = MPI_Comm_rank(COM, &id)))
    goto err;

  printf("Hello, I'm %d and there are %d of us!\n", id, pc);

  if ((errno = MPI_Finalize()))
    goto err;

  return 0;

err:
  MPI_Abort(COM, errno);
  return -1;
}
\end{minted}

\subsection{Текст командного файла}

\begin{minted}{bash}
#!/bin/bash

mpirun -n 10 --oversubscribe ./app/prll/mpi/hw
\end{minted}

\subsection{Результат выполнения}

\begin{minted}{console}
[...]$ ./run
Hello, I'm 4 and there are 10 of us!
Hello, I'm 7 and there are 10 of us!
Hello, I'm 2 and there are 10 of us!
Hello, I'm 5 and there are 10 of us!
Hello, I'm 1 and there are 10 of us!
Hello, I'm 0 and there are 10 of us!
Hello, I'm 6 and there are 10 of us!
Hello, I'm 8 and there are 10 of us!
Hello, I'm 3 and there are 10 of us!
Hello, I'm 9 and there are 10 of us!
\end{minted}

\newpage

\section{Обмен сообщениями в MPI}
\subsection{Постановка задачи}

\begin{itemize}
  \item Процессы с номером отличным от 0 генерируют вектор размера 100000 (у каждого процесса свой вектор), вычисляют его норму и посылают результат процессу с номером 0.
  \item Процесс с номером 0 распечатывает полученные значения с указанием номера процесса, приславшего сообщение.
\end{itemize}

\subsection{Текст программы}

\begin{minted}{c}
void vec_nrm(struct vec* vp, double* rp) {
  int n = vp->n;

  real* vvp = vp->v;
  preal s = 0;

#ifdef OMP_THREADS_NUM
#pragma omp parallel for reduction(+ : s) num_threads(OMP_THREADS_NUM)
#endif  // OMP
  for (int i = 0; i < n; ++i)
    s += vvp[i] * vvp[i];

  *rp = sqrt(s);
}
\end{minted}

\subsubsection{Блокирующая версия}

\begin{minted}{c}
#define N 100000

#define TAG 0
#define DBL MPI_DOUBLE
#define COM MPI_COMM_WORLD
#define ANY MPI_ANY_SOURCE
#define ROOT 0

int main(int argc, char* argv[argc]) {
  if ((errno = MPI_Init(&argc, &argv)))
    goto err;

  int pc;
  int id;

  if ((errno = MPI_Comm_size(COM, &pc)))
    goto err;

  if ((errno = MPI_Comm_rank(COM, &id)))
    goto err;

  double nrm;

  if (id == ROOT)
    for (int i = 0; i < pc - 1; ++i) {
      MPI_Status stat;

      if ((errno = MPI_Recv(&nrm, 1, DBL, ANY, TAG, COM, &stat)))
        goto err;

      trace("%d -> %lf", stat.MPI_SOURCE, nrm);
    }
  else {
    struct vec* v = vec_new(N);

    if (!v)
      goto err;

    vec_seq(v, id);
    vec_nrm(v, &nrm);

    if ((errno = MPI_Send(&nrm, 1, DBL, ROOT, TAG, COM))) {
      vec_free(v);
      goto err;
    }

    vec_free(v);
  }

  if ((errno = MPI_Finalize()))
    goto err;

  return 0;

err:
  MPI_Abort(COM, errno);
  return -1;
}
\end{minted}

\subsubsection{Неблокирующая версия}

\begin{minted}{c}
#define N 100000

#define TAG 0
#define DBL MPI_DOUBLE
#define COM MPI_COMM_WORLD
#define ROOT 0

int main(int argc, char* argv[argc]) {
  if ((errno = MPI_Init(&argc, &argv)))
    goto err;

  int pc;
  int id;

  if ((errno = MPI_Comm_size(COM, &pc)))
    goto err;

  if ((errno = MPI_Comm_rank(COM, &id)))
    goto err;

  if (id == ROOT) {
    MPI_Request* req = malloc(sizeof(MPI_Request) * (pc - 1));

    if (!req)
      goto err;

    double* nrm = malloc(sizeof(double) * (pc - 1));

    if (!nrm) {
      free(req);
      goto err;
    }

    int* f = malloc(sizeof(int) * (pc - 1));

    if (!f) {
      free(req);
      free(nrm);
      goto err;
    }

    for (int i = 0; i < pc - 1; ++i) {
      f[i] = 0;

      if ((errno = MPI_Irecv(&nrm[i], 1, DBL, i + 1, TAG, COM, &req[i]))) {
        free(req);
        free(nrm);
        free(f);

        goto err;
      }
    }

    int s = 0;

    while (s != pc - 1)
      for (int i = 0; i < pc - 1; ++i)
        if (!f[i]) {
          if ((errno = MPI_Test(&req[i], &f[i], NULL))) {
            free(req);
            free(nrm);
            free(f);

            goto err;
          }

          if (f[i]) {
            s += 1;
            trace("%d -> %lf", i + 1, nrm[i]);
          }
        }

    free(req);
    free(nrm);
    free(f);
  } else {
    srand(time(NULL) + id);

    struct vec* v = vec_new(N);

    if (!v)
      goto err;

    double nrm;

    vec_seq(v, id);
    vec_nrm(v, &nrm);

    if (id == 1)
      sleep(10);

    if ((errno = MPI_Send(&nrm, 1, DBL, ROOT, TAG, COM))) {
      vec_free(v);
      goto err;
    }

    vec_free(v);
  }

  if ((errno = MPI_Finalize()))
    goto err;

  return 0;

err:
  MPI_Abort(COM, errno);
  return -1;
}
\end{minted}

\subsection{Результат выполнения}
\subsubsection{Блокирующая версия}

\begin{minted}{console}
[...]$ mpirun -n 10 --oversubscribe ./app/prll/mpi/nrm
[2024-10-19 11:47:27]: 8 -> 18259472.581375
[2024-10-19 11:47:27]: 4 -> 18258377.106140
[2024-10-19 11:47:27]: 2 -> 18257829.376736
[2024-10-19 11:47:27]: 3 -> 18258103.240753
[2024-10-19 11:47:27]: 5 -> 18258650.972895
[2024-10-19 11:47:27]: 7 -> 18259198.710513
[2024-10-19 11:47:27]: 9 -> 18259746.453607
[2024-10-19 11:47:27]: 6 -> 18258924.841020
[2024-10-19 11:47:27]: 1 -> 18257555.514088
\end{minted}

\subsubsection{Неблокирующая версия}

\begin{minted}{console}
[...]$ mpirun -n 10 --oversubscribe ./app/prll/mpi/nrm-nb
[2024-10-19 11:47:55]: 6 -> 18258924.841020
[2024-10-19 11:47:55]: 3 -> 18258103.240753
[2024-10-19 11:47:55]: 5 -> 18258650.972895
[2024-10-19 11:47:55]: 8 -> 18259472.581375
[2024-10-19 11:47:55]: 4 -> 18258377.106140
[2024-10-19 11:47:55]: 2 -> 18257829.376736
[2024-10-19 11:47:55]: 7 -> 18259198.710513
[2024-10-19 11:47:55]: 9 -> 18259746.453607
[2024-10-19 11:48:05]: 1 -> 18257555.514088
\end{minted}

\newpage

\section{Коллективные операции в MPI}
\subsection{Постановка задачи}

Написать программу параллельного умножения матрицы на вектор с разрезанием матрицы на горизонтальные полосы неравной длины.

\subsection{Текст программы}

\begin{minted}{c}
#define TAG 0
#define ROOT 0
#define COM MPI_COMM_WORLD
#define DBL MPI_DOUBLE
#define INT MPI_INT

#define N 10

int main(int argc, char* argv[argc]) {

  if ((errno = MPI_Init(&argc, &argv))) {
    MPI_Abort(COM, errno);
    return -1;
  }

  int pc;
  int id;

  if ((errno = MPI_Comm_size(COM, &pc))) {
    MPI_Abort(COM, errno);
    return -1;
  }

  if ((errno = MPI_Comm_rank(COM, &id))) {
    MPI_Abort(COM, errno);
    return -1;
  }

  struct vec* xp = 0;
  struct vec* fp = 0;
  struct pmtx* mp = 0;

  int* t = 0;
  int* td = 0;
  int* tc = 0;

  int* vd = 0;
  int* vc = 0;

  int* fd = 0;

  if (id == ROOT) {
    t = malloc(sizeof(int) * pc);

    if (!t)
      goto end;

    td = malloc(sizeof(int) * pc);

    if (!td)
      goto end;

    tc = malloc(sizeof(int) * pc);

    if (!tc)
      goto end;

    vd = malloc(sizeof(int) * pc);

    if (!vd)
      goto end;

    vc = malloc(sizeof(int) * pc);

    if (!vc)
      goto end;

    fd = malloc(sizeof(int) * pc);

    if (!fd)
      goto end;

    for (int i = 0; i < pc; ++i) {
      t[i] = 0;
      td[i] = i;
      tc[i] = 1;
      vc[i] = 0;
      fd[i] = 0;
    }
  } else {
    t = malloc(sizeof(int));

    if (!t)
      goto end;
  }

  int p = N < pc ? N : pc;

  if (id == ROOT)
    for (int i = p - 1, s = N; i >= 0; --i) {
      t[td[i]] = s / (i + 1);
      s = s - t[td[i]];

      vc[i] = t[td[i]] * N;
      vd[i] = s * N;

      fd[i] = s;
    }

  errno = MPI_Scatterv(t, tc, td, INT, t, 1, INT, ROOT, COM);

  if (errno)
    goto end;

  mp = pmtx_new(N, 0, id == ROOT ? N : *t);

  if (!mp)
    goto end;

  if (id == ROOT)
    pmtx_seq(mp, 1);

  errno = MPI_Scatterv(mp->v, vc, vd, DBL, mp->v, *t * N, DBL, ROOT, COM);

  if (errno)
    goto end;

  if (id == ROOT) {
    mp->c = *t;
  }

  xp = vec_new(N);

  if (!xp)
    goto end;

  if (id == ROOT)
    vec_seq(xp, 1);

  errno = MPI_Bcast(xp->v, N, DBL, ROOT, COM);

  if (errno)
    goto end;

  fp = vec_new(N);

  if (!fp)
    goto end;

  if (mtx_vmlt(mp, xp, fp))
    goto end;

  errno = MPI_Gatherv(fp->v, *t, DBL, fp->v, t, fd, DBL, ROOT, COM);

  if (errno)
    goto end;

  if (id == ROOT) {
    vec_fput(stdout, fp);
    putchar('\n');
  }
end:
  ...

  if (errno) {
    MPI_Abort(COM, errno);
    printf("fatal: %s.\n", strerror(errno));
    return -1;
  }

  if ((errno = MPI_Finalize())) {
    printf("fatal: %s.\n", strerror(errno));
    MPI_Abort(COM, errno);
    return -1;
  }

  return 0;
}
\end{minted}

\subsection{Результат выполнения}

\begin{minted}{console}
[...]$ mpirun -n 1 --oversubscribe ./app/prll/mpi/mlt
5.500e+01 1.300e+02 2.050e+02 2.800e+02 3.550e+02

[...]$ mpirun -n 8 --oversubscribe ./app/prll/mpi/mlt
5.500e+01 1.300e+02 2.050e+02 2.800e+02 3.550e+02

[...]$ mpirun -n 1 --oversubscribe ./app/prll/mpi/mlt
3.850e+02 9.350e+02 1.485e+03 2.035e+03 2.585e+03 \
3.135e+03 3.685e+03 4.235e+03 4.785e+03 5.335e+03

[...]$ mpirun -n 8 --oversubscribe ./app/prll/mpi/mlt
3.850e+02 9.350e+02 1.485e+03 2.035e+03 2.585e+03 \
3.135e+03 3.685e+03 4.235e+03 4.785e+03 5.335e+03
\end{minted}

\end{document}

