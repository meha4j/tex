\documentclass[12pt, a4paper]{article}

\usepackage[
  margin=1in
]{geometry}

\usepackage{graphicx}

\usepackage[T1, T2A]{fontenc}
\usepackage[utf8]{inputenc}
\usepackage[russian]{babel}
\usepackage{textcomp}
\usepackage{minted}

\begin{document}

\begin{titlepage}
  \centering
  \textsc{Новосибирский государственный технический университет}\par
  \vspace{1mm}
  Кафедра прикладной математики\par
  \vspace{4cm}
  \textsc{Практическая работа \textnumero 1}\par
  {\huge\bfseries Основы работы с OpenMP\par}
  \vspace{1cm}
  {\scriptsize ФПМИ, ПМ-24\par}
  \vspace{1mm}
  {\itshape\large Параскун И., Шакиров П., Герасименко В.\par}
  \vfill
  {\small преподаватель\par}
  \vspace{1mm}
  \textsc{Домников Петр Александрович}
  \vfill
  \large{Новосибирск, 2024}
\end{titlepage}

\newpage

\section{Процесс выполнения}
\subsection{Пустой проект}

\begin{minted}{c}
int main() {
  return 0;
}
\end{minted}
\textit{*листинг}

\subsection{Доступное число потоков}

\begin{minted}{c}
#include <stdio.h>
#include <omp.h>

int main() {
  printf("Max threads: %d.\n", omp_get_max_threads());

  return 0;
}
\end{minted}
\textit{*листинг}

\begin{minted}{console}
[mehandes@pluto build]$ ./main
Max threads: 8.
\end{minted}
\textit{*результат выполнения}

\begin{minted}{console}
[mehandes@pluto build]$ lscpu
Architecture:             x86_64
  CPU op-mode(s):         32-bit, 64-bit
  Address sizes:          43 bits physical, 48 bits virtual
  Byte Order:             Little Endian
CPU(s):                   8
  On-line CPU(s) list:    0-7
Vendor ID:                AuthenticAMD
  Model name:             AMD Ryzen 5 3500U with Radeon Vega Mobile Gfx
    CPU family:           23
    Model:                24
    Thread(s) per core:   2
    Core(s) per socket:   4
    Socket(s):            1
    Stepping:             1
    Frequency boost:      enabled
    ...
\end{minted}
\textit{*информация о процессоре}

\subsection{Параллельная секция}

\begin{minted}{c}
#include <omp.h>
#include <stdio.h>

int main() {
#pragma omp parallel
  { printf("Hello, World!\n"); }

  return 0;
}
\end{minted}
\textit{*листинг}

\begin{minted}{console}
[mehandes@pluto build]$ ./main
Hello, World!
Hello, World!
Hello, World!
Hello, World!
Hello, World!
Hello, World!
Hello, World!
Hello, World!
\end{minted}
\textit{*результат выполнения}

\subsection{Текущее число потоков}

\begin{minted}{c}
#include <omp.h>
#include <stdio.h>

int main() {
  printf("Active threads [outside]: %d.\n", omp_get_num_threads());

#pragma omp parallel
  {
    printf("Active threads [inside]: %d.\n", omp_get_num_threads());
  }

  return 0;
}
\end{minted}
\textit{*листинг}

\begin{minted}{console}
[mehandes@pluto build]$ ./main
Active threads [outside]: 1.
Active threads [inside]: 8.
Active threads [inside]: 8.
Active threads [inside]: 8.
Active threads [inside]: 8.
Active threads [inside]: 8.
Active threads [inside]: 8.
Active threads [inside]: 8.
Active threads [inside]: 8.
\end{minted}
\textit{*результат выполнения}

\subsection{Номера потоков}

\begin{minted}{c}
#include <omp.h>
#include <stdio.h>

int main() {
#pragma omp parallel
  { printf("Hello, World [%d].\n", omp_get_thread_num()); }

  return 0;
}
\end{minted}
\textit{*листинг}

\begin{minted}{console}
[mehandes@pluto build]$ ./main
Hello, World [4].
Hello, World [0].
Hello, World [3].
Hello, World [6].
Hello, World [5].
Hello, World [7].
Hello, World [1].
Hello, World [2].
\end{minted}
\textit{*результат выполнения}

\subsection{Секции}

\begin{minted}{c}
#include <omp.h>
#include <stdio.h>

int main() {
#pragma omp parallel sections num_threads(2)
  {
#pragma omp section
    { printf("Hello, World [1; %d].\n", omp_get_thread_num()); }

#pragma omp section
    { printf("Hello, World [2; %d].\n", omp_get_thread_num()); }

#pragma omp section
    { printf("Hello, World [3; %d].\n", omp_get_thread_num()); }
  }

  return 0;
}
\end{minted}
\textit{*листинг}

\begin{minted}{console}
[mehandes@pluto build]$ ./main
Hello, World [1; 0].
Hello, World [3; 0].
Hello, World [2; 1].
\end{minted}
\textit{*результат выполнения}

\subsection{Параллельный цикл}

\begin{minted}{c}
#include <omp.h>
#include <stdio.h>

#define N 20

int main() {
#pragma omp parallel for
  for (int i = 0; i < N; ++i)
    printf("[%d; %d]\n", i, omp_get_thread_num());

  return 0;
}
\end{minted}
\textit{*листинг}

\begin{minted}{console}
[mehandes@pluto tmp]$ ./main
[0; 0]
[1; 0]
[2; 0]
[9; 3]
[10; 3]
[11; 3]
[18; 7]
[19; 7]
[16; 6]
[17; 6]
[12; 4]
[13; 4]
[6; 2]
[7; 2]
[8; 2]
[3; 1]
[4; 1]
[5; 1]
[14; 5]
[15; 5]
\end{minted}
\textit{*результат выполнения}

\subsubsection{Упорядоченность}

\begin{minted}{c}
#include <omp.h>
#include <stdio.h>

#define N 20

int main() {
  int r[N];

#pragma omp parallel for
  for (int i = 0; i < 20; ++i)
    r[i] = omp_get_thread_num();

  for (int i = 0; i < 20; ++i)
    printf("[%d; %d]\n", i, r[i]);

  return 0;
}
\end{minted}
\textit{*листинг}

\begin{minted}{console}
[mehandes@pluto tmp]$ ./a.out
[0; 0]
[1; 0]
[2; 0]
[3; 1]
[4; 1]
[5; 1]
[6; 2]
[7; 2]
[8; 2]
[9; 3]
[10; 3]
[11; 3]
[12; 4]
[13; 4]
[14; 5]
[15; 5]
[16; 6]
[17; 6]
[18; 7]
[19; 7]
\end{minted}
\
\subsection{Распределение итераций + упорядоченность}
\subsubsection{Статика}

\begin{minted}{c}
#include <omp.h>
#include <stdio.h>

#define N 20

int main() {
  int r[N];

#pragma omp parallel for schedule(static, 1)
  for (int i = 0; i < N; ++i)
    r[i] = omp_get_thread_num();

  for (int i = 0; i < N; ++i)
    printf("[%d; %d]\n", i, r[i]);

  return 0;
}
\end{minted}
\textit{*листинг}

\begin{minted}{console}
[mehandes@pluto build]$ ./main
[0; 0]
[1; 1]
[2; 2]
[3; 3]
[4; 4]
[5; 5]
[6; 6]
[7; 7]
[8; 0]
[9; 1]
[10; 2]
[11; 3]
[12; 4]
[13; 5]
[14; 6]
[15; 7]
[16; 0]
[17; 1]
[18; 2]
[19; 3]
\end{minted}
\textit{*результат выполнения, m = 1}

\begin{minted}{console}
[mehandes@pluto build]$ ./main
[0; 0]
[1; 0]
[2; 1]
[3; 1]
[4; 2]
[5; 2]
[6; 3]
[7; 3]
[8; 4]
[9; 4]
[10; 5]
[11; 5]
[12; 6]
[13; 6]
[14; 7]
[15; 7]
[16; 0]
[17; 0]
[18; 1]
[19; 1]
\end{minted}
\textit{*результат выполнения, m = 2}

\begin{minted}{console}
[mehandes@pluto build]$ ./main
[0; 0]
[1; 0]
[2; 0]
[3; 0]
[4; 0]
[5; 1]
[6; 1]
[7; 1]
[8; 1]
[9; 1]
[10; 2]
[11; 2]
[12; 2]
[13; 2]
[14; 2]
[15; 3]
[16; 3]
[17; 3]
[18; 3]
\end{minted}
\textit{*результат выполнения, m = 5}

\begin{minted}{console}
[mehandes@pluto build]$ ./main
[0; 0]
[1; 0]
[2; 0]
[3; 0]
[4; 0]
[5; 0]
[6; 0]
[7; 0]
[8; 0]
[9; 0]
[10; 1]
[11; 1]
[12; 1]
[13; 1]
[14; 1]
[15; 1]
[16; 1]
[17; 1]
[18; 1]
[19; 1]
\end{minted}
\textit{*результат выполнения, m = 10}

\subsubsection{Динамика}

\begin{minted}{c}
#include <omp.h>
#include <stdio.h>

#define N 20

int main() {
  int r[N];

#pragma omp parallel for schedule(dynamic, 1)
  for (int i = 0; i < N; ++i)
    r[i] = omp_get_thread_num();

  for (int i = 0; i < N; ++i)
    printf("[%d; %d]\n", i, r[i]);

  return 0;
}
\end{minted}
\textit{*листинг}

\begin{minted}{console}
[mehandes@pluto build]$ ./main
[0; 0]
[1; 4]
[2; 2]
[3; 7]
[4; 3]
[5; 5]
[6; 6]
[7; 1]
[8; 0]
[9; 4]
[10; 2]
[11; 6]
[12; 7]
[13; 0]
[14; 5]
[15; 1]
[16; 6]
[17; 2]
[18; 3]
[19; 7]
\end{minted}
\textit{*результат выполнения, m = 1}

\begin{minted}{console}
[mehandes@pluto build]$ ./main
[0; 3]
[1; 3]
[2; 0]
[3; 0]
[4; 6]
[5; 6]
[6; 5]
[7; 5]
[8; 4]
[9; 4]
[10; 2]
[11; 2]
[12; 7]
[13; 7]
[14; 6]
[15; 6]
[16; 3]
[17; 3]
[18; 0]
[19; 0]
\end{minted}
\textit{*результат выполнения, m = 2}

\begin{minted}{console}
[0; 3]
[1; 3]
[2; 3]
[3; 3]
[4; 3]
[5; 7]
[6; 7]
[7; 7]
[8; 7]
[9; 7]
[10; 4]
[11; 4]
[12; 4]
[13; 4]
[14; 4]
[15; 0]
[16; 0]
[17; 0]
[18; 0]
[19; 0]
\end{minted}
\textit{*результат выполнения, m = 5}

\begin{minted}{console}
[0; 4]
[1; 4]
[2; 4]
[3; 4]
[4; 4]
[5; 4]
[6; 4]
[7; 4]
[8; 4]
[9; 4]
[10; 0]
[11; 0]
[12; 0]
[13; 0]
[14; 0]
[15; 0]
[16; 0]
[17; 0]
[18; 0]
[19; 0]
\end{minted}
\textit{*результат выполнения, m = 10}

\end{document}

