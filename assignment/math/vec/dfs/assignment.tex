\documentclass[12pt, a4paper]{article}

\usepackage[
  margin=1in
]{geometry}

\usepackage[T1, T2A]{fontenc}
\usepackage[utf8]{inputenc}
\usepackage[russian]{babel}
\usepackage{minted}
\usepackage{hyperref}
\usepackage{tabularray}
\usepackage{amsmath}
\usepackage{float}

\begin{document}

\begin{titlepage}
  \centering
  \textsc{Новосибирский государственный технический университет}\par
  \vspace{1mm}
  Кафедра прикладной математики\par
  \vspace{4cm}
  \textsc{Практическая работа \textnumero 1}\par
  {\huge\bfseries Прямые методы решения СЛАУ\par}
  \vspace{1cm}
  {\scriptsize ФПМИ, ПМ-24\par}
  \vspace{1mm}
  {\itshape\large Параскун И., Герасименко В.\par}
  \vfill
  {\small преподаватели\par}
  \vspace{2mm}
  \textsc{Рояк М. Э., д.т.н., профессор}\par
  \vspace{1mm}
  \textsc{Задорожный А. Г., к.т.н., доцент}\par
  \vspace{1mm}
  \textsc{Леонович Д. А., ассистент}\par
  \vfill
  \large{Новосибирск, 2024}
\end{titlepage}

\newpage

\setcounter{page}{2}
\tableofcontents

\newpage

\section{Метод LDU разложения}
\subsection{Постановка задачи}
Реализовать метод LDU разложения с учетом следующих требований:

\begin{itemize}
  \item Матрица задается и хранится в профильном формате.
  \item Элементы матрицы обрабатывать в порядке, соответствующем 
    формату хранения.
  \item Предусмотреть возможность смены точности представления чисел.
\end{itemize}

\subsection{Исходный код}

\begin{minted}{c}
struct mtx {
  real* d;
  real* l;
  real* u;

  int* p;

  int n;
  int s;
};

void smtx_ldu(struct smtx* mp) {
  int n = mp->n;

  int* mpp = mp->p;
  real* mdp = mp->d;
  real* mlp = mp->l;
  real* mup = mp->u;

  for (int i = 0; i < n; ++i) {
    int ic = mpp[i + 1] - mpp[i];
    preal dsum = mdp[i];

    for (int k = 1; k <= ic; ++k)
      dsum -= mlp[mpp[i] + ic - k] * mup[mpp[i] + ic - k] * mdp[i - k];

    mdp[i] = dsum;

    for (int j = i + 1; j < n; ++j) {
      int jc = mpp[j + 1] - mpp[j] - j + i;

      if (jc > -1) {
        preal lsum = mlp[mpp[j] + jc];
        preal usum = mup[mpp[j] + jc];

        for (int k = 1; k <= min(jc, ic); ++k) {
          lsum -= mlp[mpp[j] + jc - k] * mup[mpp[i] + ic - k] * mdp[i - k];
          usum -= mup[mpp[j] + jc - k] * mlp[mpp[i] + ic - k] * mdp[i - k];
        }

        mlp[mpp[j] + jc] = lsum / mdp[i];
        mup[mpp[j] + jc] = usum / mdp[i];
      }
    }
  }
}

static void ssle_l(struct smtx* a, struct vec* y, struct vec* b) {
  int n = y->n;
  int* pp = a->p;

  real* lp = a->l;
  real* yp = y->v;
  real* bp = b->v;

  for (int i = 0; i < n; ++i) {
    preal sum = 0;

    int k0 = pp[i];
    int k1 = pp[i + 1];
    int j0 = i - k1 + k0;

    for (int k = k0, j = j0; k < k1; ++k, ++j)
      sum += lp[k] * yp[j];

    yp[i] = bp[i] - sum;
  }
}

static void ssle_d(struct smtx* a, struct vec* y, struct vec* b) {
  int n = y->n;

  real* dp = a->d;
  real* yp = y->v;
  real* bp = b->v;

  for (int i = 0; i < n; ++i) {
    preal e = bp[i] / dp[i];
    yp[i] = e;
  }
}

static void ssle_u(struct smtx* a, struct vec* y, struct vec* b) {
  int n = y->n;
  int* pp = a->p;

  real* up = a->u;
  real* yp = y->v;
  real* bp = b->v;

  memcpy(yp, bp, sizeof(real) * n);

  for (int j = n - 1; j > 0; --j) {
    int k0 = pp[j];
    int k1 = pp[j + 1];
    int i0 = j - k1 + k0;

    for (int k = k0, i = i0; k < k1; ++k, ++i)
      yp[i] -= up[k] * yp[j];
  }
}

void ssle_ldu(struct smtx* a, struct vec* x, struct vec* b) {
  smtx_ldu(a);

  ssle_l(a, x, b);
  ssle_d(a, b, x);
  ssle_u(a, x, b);
}
\end{minted}

\subsection{Исследование}
\subsubsection{Матрица с диагональным преобладанием}

\begin{table}[H]
\centering
\begin{tblr}{
  width=\textwidth, 
  colspec={|l|X|X|X|X|X|X|},
  rowspec={|c|c|lllllllllllllllll|}
}
\SetCell[r=2]{c} $k$  & \SetCell[c=3]{c} $||x^k||$              &&& \SetCell[c=3]{c} $||x^*-x^k||$        &&  \\
                      & 32 - 32     & 64 - 64     & 64 - 32     & 32 - 32     & 64 - 64     & 64 - 32         \\
1	                    & 1.9621E+01	& 1.9621E+01	& 1.9622E+01	& 2.6575E-05	& 1.0893E-12	& 2.4138E-04      \\ 
2	                    & 1.9615E+01	& 1.9621E+01	& 1.9624E+01	& 7.5095E-03	& 6.1424E-12	& 2.7088E-03      \\
3	                    & 1.9645E+01	& 1.9621E+01	& 1.9581E+01	& 2.6272E-02	& 1.1746E-10	& 4.5066E-02      \\
4	                    & 1.9440E+01	& 1.9621E+01	& 1.9629E+01	& 2.0455E-01	& 3.3541E-09	& 8.4352E-03      \\
5	                    & 1.2640E+01	& 1.9621E+01	& 1.6017E+01	& 8.6026E+00	& 3.9932E-09	& 4.2005E+00      \\
6	                    & 1.2326E+01	& 1.9621E+01	& 1.4998E+01	& 2.5724E+01	& 1.3900E-07	& 5.4571E+00      \\
7	                    & 1.0312E+01	& 1.9621E+01	& 1.1073E+01	& 2.2275E+01	& 6.6127E-07	& 1.1059E+01      \\
8	                    & 1.0009E+01	& 1.9621E+01	& 1.6261E+01	& 2.1597E+01	& 1.4153E-05	& 3.0881E+01      \\
9	                    & 2.8423E+01	& 1.9622E+01	& 9.4575E+00	& 9.5404E+00	& 1.3386E-04	& 1.4757E+01      \\
10	                  & 9.3097E+00	& 1.9621E+01	& 2.2859E+01	& 1.5351E+01	& 6.1039E-04	& 3.5849E+00      \\
11	                  & 1.6882E+01	& 1.9622E+01	& 9.8977E+00	& 3.1623E+01	& 1.1783E-03	& 1.3460E+01      \\
12	                  & 2.9443E+01	& 1.9754E+01	& 2.0667E+01	& 1.0614E+01	& 1.4896E-01	& 1.1712E+00      \\
13	                  & 1.6882E+01	& 1.9721E+01	& 2.7092E+01	& 3.1623E+01	& 1.1270E-01	& 8.1316E+00      \\
14	                  & 9.1513E+00	& 2.4144E+03	& 1.2078E+01	& 1.6276E+01	& 2.3970E+03	& 9.4314E+00      \\
15	                  & 9.1518E+00	& 3.5524E+01	& 9.5654E+00	& 1.8513E+01	& 5.1736E+01	& 1.4393E+01      \\
16	                  & 9.1724E+00	& 1.6882E+01	& 3.1097E+01	& 1.8670E+01	& 3.1623E+01	& 1.2348E+01      \\
17	                  & 2.8860E+01	& 1.0180E+01	& inf	        & 4.4786E+01	& 1.2796E+01	& inf             \\
\end{tblr}
\end{table}

$\sum_{j=1}^{n}a_{0j} \approx -31$ \\

\noindentВведение погрешности через элемент $a_{00}$ обеспечивает невырожденность
матрицы. В случае, когда погрешность вводится в порядок, близкий к числу
значимых цифр исследуемого типа, матрица становится вырожденной.

\subsubsection{Матрица Гильберта}

\begin{table}[H]
\centering
\begin{tblr}{
  width=\textwidth, 
  colspec={|l|X|X|X|X|X|X|},
  rowspec={|c|c|lllllllllllllll|}
}
\SetCell[r=2]{c} $k$  & \SetCell[c=3]{c} $||x^k||$              &&& \SetCell[c=3]{c} $||x^*-x^k||$        &&  \\
                      & 32 - 32     & 64 - 64     & 64 - 32     & 32 - 32     & 64 - 64     & 64 - 32         \\
3	                    & 3.7417E+00	& 3.7417E+00	& 3.7417E+00	& 1.4573E-05	& 1.2586E-14	& 6.7866E-06      \\
4	                    & 5.4772E+00	& 5.4772E+00	& 5.4771E+00	& 9.6918E-04	& 1.8041E-12	& 1.3854E-03      \\
5	                    & 7.4165E+00	& 7.4162E+00	& 7.4173E+00	& 2.1355E-02	& 2.7379E-11	& 5.8729E-02      \\
6	                    & 9.5557E+00	& 9.5394E+00	& 9.5399E+00	& 5.1900E-01	& 5.4220E-10	& 3.1442E-02      \\
7	                    & 4.4204E+01	& 1.1832E+01	& 2.4900E+01	& 4.2582E+01	& 5.0860E-08	& 2.1917E+01      \\
8	                    & 2.8041E+02	& 1.4283E+01	& 7.0833E+01	& 2.8005E+02	& 4.3366E-06	& 6.9373E+01      \\
9	                    & 5.4545E+02	& 1.6882E+01	& 1.8302E+02	& 5.4518E+02	& 3.6263E-04	& 1.8224E+02      \\
10	                  & 1.8713E+02	& 1.9621E+01	& 2.7851E+02	& 1.8609E+02	& 9.9118E-03	& 2.7782E+02      \\
11	                  & 5.4125E+01	& 2.2500E+01	& 2.0314E+02	& 4.9233E+01	& 4.8656E-01	& 2.0189E+02      \\
12	                  & 3.1538E+02	& 2.6002E+01	& 5.5247E+02	& 3.1435E+02	& 5.1111E+00	& 5.5187E+02      \\
13	                  & 1.2686E+02	& 5.2931E+01	& 4.5245E+02	& 1.2360E+02	& 4.4527E+01	& 4.5154E+02      \\
14	                  & 2.2309E+02	& 4.7094E+02	& 5.5835E+02	& 2.2080E+02	& 4.6986E+02	& 5.5743E+02      \\
15	                  & 3.5406E+02	& 1.6523E+02	& 4.5408E+02	& 3.5230E+02	& 1.6144E+02	& 4.5270E+02      \\
16	                  & 6.5417E+02	& 6.1479E+02	& 7.5275E+02	& 6.5301E+02	& 6.1357E+02	& 7.5175E+02      \\
17	                  & 2.8344E+02	& 2.9924E+02	& 8.6594E+02	& 2.8026E+02	& 2.9624E+02	& 8.6491E+02      \\

\end{tblr}
\end{table}

Число обусловленности матрицы Гильберта возрастает 
как $\mathcal{O}((1+\sqrt{2})^{4n}/\sqrt{n})$, что объясняет быстрый рост
погрешности результата с ростом погрешности входных данных.

\subsection{Анализ}
\subsubsection{Расчет количества действий}

\begin{gather}
  d_{ii} = a_{ii} - \sum_{j=1}^{i-1}l_{ij}d_{jj}u_{ji}; i \in [1, n] \\
  l_{ij} = \frac{a_{ij} - \sum_{k=1}^{j-1}l_{ik}d_{kk}u_{kj}}{d_{jj}}; i, j \in [1, n], i \ne j \\
  u_{ij} = \frac{a_{ij} - \sum_{k=1}^{i-1}l_{ik}d_{kk}u_{kj}}{d_{ii}}; i, j \in [1, n], i \ne j \\
  N_d = \sum_{i=1}^{n}3(i-1) = \frac{3n(n-1)}{2} \\
  N_l = N_u = \sum_{j=1}^{n}(n-j)[3(j-1)+1] = \frac{n^3 - 2n^2 + n}{2} \\
  N_{ldu} = n^3 - \frac{1}{2}n^2 - \frac{1}{2}n \\
  x_i^l = f_i - \sum_{j=1}^{i-1}l_{ij}x_j; i \in [1, n] \\
  N_l = N_u = \sum_{i=1}^{n}2(i-1) = n(n-1) \\
  x_i^d = \frac{f_i}{d_{ii}}; i \in [1, n] \\
  N_d = n \\
  N_{rm} = 2n(n-1) + n = n(2n-1) \\
  N_{ldu} + N_{rm} = \frac{2n^3 + 3n^2 - 3n}{2}
\end{gather}

\newpage

\section{Метод Гаусса с выбором ведущего элемента}
\subsection{Постановка задачи}
Реализовать метод Гаусса с выбором ведущего элемента с учетом следующих требований:

\begin{itemize}
  \item Матрица задается и хранится в плотном формате.
\end{itemize}

\subsection{Исходный код}

\begin{minted}{c}
struct mtx {
  real* v;
  int n;
};

static inline void swap(int* p, int a, int b) {
  if (a == b)
    return;

  int tmp = p[a];
  p[a] = p[b];
  p[b] = tmp;
}

void dsle_gauss(struct dmtx* mp, struct vec* yp, struct vec* bp) {
  int n = mp->n;
  int* pp = malloc(sizeof(int) * n);

  real* mvp = mp->v;
  real* yvp = yp->v;
  real* bvp = bp->v;

  for (int i = 0; i < n; ++i)
    pp[i] = i;

  for (int i = 0; i < n; ++i) {
    real max = fabs(mvp[pp[i] * n + i]);
    int maxi = i;

    for (int j = i + 1; j < n; ++j) {
      real mij = fabs(mvp[pp[j] * n + i]);

      if (mij > max) {
        max = mij;
        maxi = j;
      }
    }

    swap(pp, i, maxi);

    int ppin = pp[i] * n;

    for (int j = i + 1; j < n; ++j) {
      int ppjn = pp[j] * n;
      preal k = mvp[ppjn + i] / mvp[ppin + i];

      for (int c = i + 1; c < n; ++c)
        mvp[ppjn + c] -= mvp[ppin + c] * k;

      bvp[pp[j]] -= bvp[pp[i]] * k;
    }
  }

  for (int o = 0, i = n - 1; o < n; ++o, --i) {
    preal sum = bvp[pp[i]];
    int ppin = pp[i] * n;

    for (int j = i + 1; j < n; ++j)
      sum -= yvp[j] * mvp[ppin + j];

    yvp[i] = sum / mvp[ppin + i];
  }

  free(pp);
}
\end{minted}

\subsection{Исследование}
\subsubsection{Матрица с диагональным преобладанием}

\begin{table}[H]
\centering
\begin{tblr}{
  width=\textwidth, 
  colspec={|l|X|X|X|X|X|X|},
  rowspec={|c|c|llllllllllllllllll|}
}
\SetCell[r=2]{c} $k$  & \SetCell[c=3]{c} $||x^k||$              &&& \SetCell[c=3]{c} $||x^*-x^k||$        &&  \\
                      & 32 - 32     & 64 - 64     & 64 - 32     & 32 - 32     & 64 - 64     & 64 - 32         \\
1	                    & 1.9621E+01	& 1.9621E+01	& 1.9621E+01	& 4.0841E-05	& 2.4939E-12	& 1.8961E-06      \\
2	                    & 1.9622E+01	& 1.9621E+01	& 1.9619E+01	& 1.0084E-03	& 9.3750E-13	& 2.4947E-03      \\
3	                    & 1.9648E+01	& 1.9621E+01	& 1.9676E+01	& 2.9857E-02	& 1.8481E-11	& 6.1410E-02      \\
4	                    & 1.9643E+01	& 1.9621E+01	& 1.9791E+01	& 2.4139E-02	& 1.8055E-09	& 1.9102E-01      \\
5	                    & 1.6440E+01	& 1.9621E+01	& 1.4282E+01	& 3.6893E+00	& 1.9753E-08	& 6.3712E+00      \\
6	                    & 1.0247E+01	& 1.9621E+01	& 1.0819E+01	& 1.2649E+01	& 1.8100E-07	& 2.3271E+01      \\
7	                    & 1.0436E+01	& 1.9621E+01	& 1.8224E+01	& 2.2531E+01	& 9.9678E-07	& 3.3191E+01      \\
8	                    & 9.6177E+00	& 1.9621E+01	& 1.0635E+01	& 1.4230E+01	& 1.1256E-05	& 2.2925E+01      \\
9	                    & 1.1377E+01	& 1.9621E+01	& 9.0882E+00	& 2.4244E+01	& 1.1948E-04	& 1.7085E+01      \\
10	                  & 1.4318E+01	& 1.9621E+01	& 2.2866E+01	& 2.8460E+01	& 6.4355E-15	& 3.5923E+00      \\
11	                  & 2.2472E+01	& 1.9629E+01	& 1.7445E+01	& 3.7947E+01	& 8.5143E-03	& 2.4990E+00      \\
12	                  & 1.5572E+01	& 1.9609E+01	& 1.8013E+01	& 3.0042E+01	& 1.4261E-02	& 3.2948E+01      \\
13	                  & 1.9621E+01	& 1.8632E+01	& 1.0296E+01	& 2.7522E-06	& 1.1244E+00	& 2.2241E+01      \\
14	                  & 1.2514E+01	& 2.8373E+01	& 1.0355E+01	& 2.6001E+01	& 9.4868E+00	& 2.2365E+01      \\
15	                  & nan	        & 1.1377E+01	& 5.4678E+01	& nan	        & 2.4244E+01	& 7.1311E+01      \\
16	                  & nan	        & 1.0247E+01	& 1.7199E+01	& nan	        & 1.2649E+01	& 3.1997E+01      \\
17	                  & nan	        & 3.3408E+01	& 1.1704E+01	& nan	        & 1.4757E+01	& 1.0011E+01      \\
18	                  & nan	        & nan	        & 1.9427E+01	& nan	        & nan	        & 3.4565E+01      \\

\end{tblr}
\end{table}

$\sum_{j=1}^{n}a_{0j} \approx -27$ \\

\noindentВведение погрешности через элемент $a_{00}$ обеспечивает невырожденность
матрицы. В случае, когда погрешность вводится в порядок, близкий к числу
значимых цифр исследуемого типа, матрица становится вырожденной.

\subsubsection{Матрица Гильберта}

\begin{table}[H]
\centering
\begin{tblr}{
  width=\textwidth, 
  colspec={|l|X|X|X|X|X|X|},
  rowspec={|c|c|llllllllllllllll|}
}
\SetCell[r=2]{c} $k$  & \SetCell[c=3]{c} $||x^k||$              &&& \SetCell[c=3]{c} $||x^*-x^k||$        &&  \\
                      & 32 - 32     & 64 - 64     & 64 - 32     & 32 - 32     & 64 - 64     & 64 - 32         \\
3	                    & 3.7417E+00	& 3.7417E+00	& 3.7417E+00	& 6.7866E-06	& 1.5444E-14	& 4.5362E-06      \\
4	                    & 5.4772E+00	& 5.4772E+00	& 5.4772E+00	& 7.6094E-04	& 1.2850E-12	& 7.7807E-04      \\
5	                    & 7.4162E+00	& 7.4162E+00	& 7.4160E+00	& 4.6073E-04	& 9.1032E-12	& 1.6090E-02      \\
6	                    & 9.6107E+00	& 9.5394E+00	& 9.5393E+00	& 1.1369E+00	& 4.2953E-10	& 4.7172E-02      \\
7	                    & 1.5630E+01	& 1.1832E+01	& 2.5899E+01	& 1.0219E+01	& 5.6940E-09	& 2.3029E+01      \\
8	                    & 2.6933E+01	& 1.4283E+01	& 2.5906E+02	& 2.2828E+01	& 4.5190E-07	& 2.5867E+02      \\
9	                    & 2.6690E+01	& 1.6882E+01	& 6.7567E+01	& 2.0672E+01	& 8.3310E-05	& 6.5423E+01      \\
10	                  & 8.1537E+01	& 1.9621E+01	& 1.2939E+02	& 7.9137E+01	& 2.9981E-03	& 1.2789E+02      \\
11	                  & 2.8857E+02	& 2.2494E+01	& 1.7046E+02	& 2.8768E+02	& 2.4259E-02	& 1.6897E+02      \\
12	                  & 7.3442E+02	& 2.5591E+01	& 1.4766E+02	& 7.3398E+02	& 2.2077E+00	& 1.4545E+02      \\
13	                  & 1.1702E+03	& 7.3355E+01	& 3.2671E+02	& 1.1699E+03	& 6.7542E+01	& 3.2545E+02      \\
14	                  & 2.1075E+04	& 1.9441E+03	& 3.8227E+02	& 2.1075E+04	& 1.9438E+03	& 3.8094E+02      \\
15	                  & 5.8574E+02	& 2.2534E+02	& 2.4898E+02	& 5.8468E+02	& 2.2257E+02	& 2.4647E+02      \\
16	                  & 3.3607E+02	& 1.7520E+02	& 5.2371E+02	& 3.3385E+02	& 1.7088E+02	& 5.2228E+02      \\
17	                  & 4.2766E+02	& 6.6211E+01	& 5.3428E+02	& 4.2557E+02	& 5.0979E+01	& 5.3260E+02      \\
18	                  & 7.2599E+02	& 1.0506E+04	& 2.2618E+02	& 7.2454E+02	& 1.0506E+04	& 2.2146E+02      \\

\end{tblr}
\end{table}

Число обусловленности матрицы Гильберта возрастает 
как $\mathcal{O}((1+\sqrt{2})^{4n}/\sqrt{n})$, что объясняет быстрый рост
погрешности результата с ростом погрешности входных данных.

\subsection{Анализ}
\subsubsection{Расчет количества действий}

\begin{gather}
  N_c = \sum_{i=1}^{n}(n-i) = n^2 - \frac{1+n}{2}n \\
  N_i = \sum_{k=i}^{n}(2 + \sum_{j=i}^{n}2) = (n-i+1)(2n-2i+3) \\
  N_t = \sum_{i=2}^{n}N_i = \frac{2}{3}n^3-\frac{2}{3}n \\
  x_i = \frac{f_i - \sum_{j=i+1}^{n}a_{ij}x_j}{a_{ii}} \\
  N_s = \sum_{i=1}^{n}(2n-2i+1) = n^2 \\
  N_{gauss} = \frac{2}{3}n^3 + n^2 -\frac{2}{3}n
\end{gather}

\section{Анализ LU разложения}
\subsection{Расчет количества действий}

\begin{gather}
  l_{ij} = a_{ij} - \sum_{k=1}^{j-1}l_{ik}u_{kj} \\
  u_{ij} = \frac{a_{ij} - \sum_{k=1}^{i-1}l_{ik}u_{kj}}{l_{ii}} \\
  N_l = \sum_{j=1}^{n}(n-j+1)2(j-1) = \frac{n^3-n}{3} \\
  N_u = \sum_{i=1}^{n-1}(n-i)[2(i-1)+1] = \frac{n^3}{3}-\frac{n^2}{2}+\frac{n}{6} \\
  N_{lu} = \frac{2n^3}{3}-\frac{n^2}{2}-\frac{n}{6} \\
  N_{rml} = n^2 \\
  N_{rmu} = n(n-1) \\
  N_{rm} = 2n^2 - n
\end{gather}

\section{Сравнение методов}

\begin{table}[H]
\centering
\begin{tblr}{
  width=\textwidth, 
  colspec={|X|X|X|X|X|X|},
  rowspec={|c|c|c|}
}
\SetCell[c=2]{c} $LDU$                      &&  \SetCell[c=2]{c} $LU$                                 &&  \SetCell[c=2]{c} $Gauss$                  &  \\
Разложение          & Решение               &   Разложение                     & Решение              &   Разложение          & Решение                \\
$\mathcal{O}(n^3)$  & $\mathcal{O}(2n^2)$   &   $\mathcal{O}(\frac{2}{3}n^3)$  & $\mathcal{O}(2n^2)$  &   $\mathcal{O}(\frac{2}{3}n^3)$  & $\mathcal{O}(n^2)$                   


\end{tblr}
\end{table}

\end{document}

