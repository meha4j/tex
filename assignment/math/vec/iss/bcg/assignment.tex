\documentclass[12pt, a4paper]{article}

\usepackage[
  margin=1in
]{geometry}

\usepackage[T1, T2A]{fontenc}
\usepackage[utf8]{inputenc}
\usepackage[russian]{babel}
\usepackage{minted}
\usepackage{hyperref}
\usepackage{tabularray}
\usepackage{amsmath}
\usepackage{float}
\usepackage{algorithm}
\usepackage{algpseudocode}

\begin{document}

\begin{titlepage}
  \centering
  \textsc{Новосибирский государственный технический университет}\par
  \vspace{1mm}
  Кафедра прикладной математики\par
  \vspace{4cm}
  \textsc{Практическая работа \textnumero 3}\par
  {\huge\bfseries Стабилизированный метод бисопряжённых градиентов\par}
  \vspace{1cm}
  {\scriptsize ФПМИ, ПМ-24\par}
  \vspace{1mm}
  {\itshape\large Параскун И., Герасименко В.\par}
  \vfill
  {\small преподаватели\par}
  \vspace{2mm}
  \textsc{Рояк М. Э., д.т.н., профессор}\par
  \vspace{1mm}
  \textsc{Задорожный А. Г., к.т.н., доцент}\par
  \vspace{1mm}
  \textsc{Леонович Д. А., ассистент}\par
  \vfill
  \large{Новосибирск, 2024}
\end{titlepage}

\newpage

\setcounter{page}{2}
\tableofcontents

\newpage

\section{Постановка задачи}
Реализовать стабилизированный метод бисопряжённых градиентов для матриц в строчно-столбцовом 
формате с учётом следующих требований:

\begin{itemize}
  \item Матрица должна обрабатываться в соответствии с форматом;
  \item Выход из итерационного процесса выполнять, если относительная 
    невязка стала меньше заданного параметра;
  \item Предусмотреть аварийный выход из итерационного процесса при достижении 
    максимального количества итераций;
\end{itemize}

\section{Стабилизированный метод бисопряжённых градиентов}
\subsection{Алгоритм}

\begin{algorithm}[H]
\caption{Стабилизированный метод бисопряжённых градиентов}\label{alg:bcgstab}
\begin{algorithmic}
\State $r_0 = b - Ax_0$
\State $\hat{r}_0 = r_0$
\State $p_0 = r_0$

\For{$j=0, 1, ...$}
\State $\alpha_j = (r_j, \hat{r}_0)/(Ap_j, \hat{r}_0)$
\State $s_j = r_j - \alpha_j Ap_j$
\State $\omega_j = (As_j, s_j)/(As_j, As_j)$
\State $x_{j+1} = x_j + \alpha_j p_j + \omega_j s_j$
\State $r_{j+1} = s_j - \omega_j As_j$

\If{$||r_{j+1}|| < \varepsilon$}
\State complete
\EndIf

\State $\beta_j = (r_{j+1}, \hat{r}_0) / (r_j, \hat{r}_0) * \alpha_j / \omega_j$
\State $p_{j+1} = r_{j+1} + \beta_j(p_j - \omega_j Ap_j)$
\EndFor
\end{algorithmic}
\end{algorithm}

\subsection{Исходный код}

\inputminted[firstline=5, lastline=78]{c}{/home/mehandes/c/src/github.com/meha4j/math/vec/iss/src/iss_csj.c}

\section{Стабилизированный метод бисопряжённых градиентов c предобуславливанием}
\subsection{Алгоритм}

\begin{algorithm}[H]
\caption{Стабилизированный метод бисопряжённых градиентов с правым предобуславливанием}\label{alg:bcgstabcon}
\begin{algorithmic}
\State $r_0 = b - Ax_0$
\State $\hat{r}_0 = r_0$
\State $p_0 = r_0$

\For{$j=0, 1, ...$}
\State $\hat{p}_j = M^{-1}p_j$
\State $\alpha_j = (r_j, \hat{r}_0)/(A\hat{p}_j, \hat{r}_0)$
\State $s_j = r_j - \alpha_j A\hat{p}_j$
\State $\hat{s}_j = M^{-1}s_j$
\State $\omega_j = (A\hat{s}_j, s_j)/(A\hat{s}_j, A\hat{s}_j)$
\State $x_{j+1} = x_j + \alpha_j \hat{p}_j + \omega_j \hat{s}_j$
\State $r_{j+1} = s_j - \omega_j A\hat{s}_j$

\If{$||r_{j+1}|| < \varepsilon$}
\State complete
\EndIf

\State $\beta_j = (r_{j+1}, \hat{r}_0) / (r_j, \hat{r}_0) * \alpha_j / \omega_j$
\State $p_{j+1} = r_{j+1} + \beta_j(p_j - \omega_j A\hat{p}_j)$
\EndFor
\end{algorithmic}
\end{algorithm}

\subsection{Исходный код}

\inputminted[firstline=80, lastline=232]{c}{/home/mehandes/c/src/github.com/meha4j/math/vec/iss/src/iss_csj.c}

\section{Исследование}
\subsection{Матрица со значительным диагональным преобладанием}

\begin{table}[H]
\centering
\begin{tblr}{
  width=\textwidth, 
  colspec={|X|X|X|X|c|X|},
  rowspec={|c|c|}
}
\SetCell{c} $||\textbf{x}||$  & \SetCell{c} $||\textbf{x}^*||$ & \SetCell{c} $||\textbf{x}^* - \textbf{x}||$ & \SetCell{c} $||\textbf{r}||$ & \SetCell{c} $N$ & \SetCell{c} $t, ms$ \\
1.9621e+01	                  & 1.9621e+01	                   & 0.0000e+00	                                 & 1.7985e-19	                  & 10	            & 0.321886
\end{tblr}
\caption{Результат работы алгоритма без предобуславливания}
\end{table}

\begin{table}[H]
\centering
\begin{tblr}{
  width=\textwidth, 
  colspec={|X|X|X|X|c|X|},
  rowspec={|c|c|}
}
\SetCell{c} $||\textbf{x}||$  & \SetCell{c} $||\textbf{x}^*||$ & \SetCell{c} $||\textbf{x}^* - \textbf{x}||$ & \SetCell{c} $||\textbf{r}||$ & \SetCell{c} $N$ & \SetCell{c} $t, ms$ \\
1.9599e+01	                  & 1.9621e+01	                   & 2.2619e-02	                                 & 6.4436e-16	                  & 90	            & 3.150197
\end{tblr}
\caption{Результат работы алгоритма с ILU предобуславливанием}
\end{table}

\begin{table}[H]
\centering
\begin{tblr}{
  width=\textwidth, 
  colspec={|X|X|X|X|c|X|},
  rowspec={|c|c|}
}
\SetCell{c} $||\textbf{x}||$  & \SetCell{c} $||\textbf{x}^*||$ & \SetCell{c} $||\textbf{x}^* - \textbf{x}||$ & \SetCell{c} $||\textbf{r}||$ & \SetCell{c} $N$ & \SetCell{c} $t, ms$ \\
1.9661e+01	                  & 1.9621e+01	                   & 3.9812e-02	                                 & 7.6221e-16	                  & 47	            & 1.544534
\end{tblr}
\caption{Результат работы алгоритма с диагональным предобуславливанием}
\end{table}

\newpage

\subsection{Матрицы Гильберта}

\begin{table}[H]
\centering
\begin{tblr}{
  width=\textwidth, 
  colspec={|c|X|X|X|X|c|X|},
  rowspec={|c|ccccccc|}
}
\SetCell{c} $k$ & \SetCell{c} $||\textbf{x}||$  & \SetCell{c} $||\textbf{x}^*||$  & \SetCell{c} $||\textbf{x}^* - \textbf{x}||$ & \SetCell{c} $||\textbf{r}||$  & \SetCell{c} $N$ & \SetCell{c} $t, ms$ \\
3               & 3.7416e+00	                  & 3.7416e+00	                    & 3.1086e-15	                                & 1.3270e-18	                  & 4	              & 0.328823            \\
4               & 5.4772e+00	                  & 5.4772e+00	                    & 1.6881e+01	                                & 1.0222e-16	                  & 6	              & 0.233018            \\
5               & 7.4161e+00	                  & 7.4161e+00	                    & 1.4974e-12	                                & 4.0491e-16	                  & 8	              & 0.308553            \\
6               & 9.5393e+00	                  & 9.5393e+00	                    & 5.5759e-12	                                & 1.7839e-16	                  & 12	            & 0.508497            \\
7               & 1.1832e+01	                  & 1.1832e+01	                    & 1.9718e-10	                                & 8.1263e-16	                  & 32	            & 1.314948            \\
8               & 1.4282e+01	                  & 1.4282e+01	                    & 1.5787e-09	                                & 9.6661e-16	                  & 720	            & 27.162724           \\
9               & 1.6881e+01	                  & 1.6881e+01	                    & 5.3014e-09	                                & 8.4781e-16	                  & 58	            & 1.823285
\end{tblr}
\caption{Результат работы алгоритма без предобуславливания}
\end{table}

\begin{table}[H]
\centering
\begin{tblr}{
  width=\textwidth, 
  colspec={|c|X|X|X|X|c|X|},
  rowspec={|c|ccccccc|}
}
\SetCell{c} $k$ & \SetCell{c} $||\textbf{x}||$  & \SetCell{c} $||\textbf{x}^*||$  & \SetCell{c} $||\textbf{x}^* - \textbf{x}||$ & \SetCell{c} $||\textbf{r}||$  & \SetCell{c} $N$ & \SetCell{c} $t, ms$ \\
3               & 3.7417e+00	                  & 3.7417e+00	                    & 1.0658e-14	                                & 1.6373e-16	                  & 1               & 0.031689            \\
4               & -	                            & -	                              & -	                                          & -	                            & -	              & -                   \\
5               & 7.4162e+00	                  & 7.4162e+00	                    & 1.5543e-13	                                & 3.0220e-16	                  & 1	              & 0.030217            \\
6               & 9.5394e+00	                  & 9.5394e+00	                    & 2.9488e-13	                                & 4.1796e-16	                  & 1	              & 0.030217            \\
7               & 1.1832e+01	                  & 1.1832e+01	                    & 1.0085e-10	                                & 2.6456e-16	                  & 2	              & 0.063329            \\
8               & 1.4283e+01	                  & 1.4283e+01	                    & 1.0148e-09	                                & 4.6718e-16	                  & 2	              & 0.063238            \\
9               & 1.6882e+01	                  & 1.6882e+01	                    & 8.7433e-09	                                & 5.3738e-16	                  & 2	              & 0.066054
\end{tblr}
\caption{Результат работы алгоритма с ILU предобуславливанием}
\end{table}

\begin{table}[H]
\centering
\begin{tblr}{
  width=\textwidth, 
  colspec={|c|X|X|X|X|c|X|},
  rowspec={|c|ccccccc|}
}
\SetCell{c} $k$ & \SetCell{c} $||\textbf{x}||$  & \SetCell{c} $||\textbf{x}^*||$  & \SetCell{c} $||\textbf{x}^* - \textbf{x}||$ & \SetCell{c} $||\textbf{r}||$  & \SetCell{c} $N$ & \SetCell{c} $t, ms$ \\
3               & 4.7816e+00	                  & 3.7417e+00	                    & 1.0399e+00	                                & 9.9973e-16	                  & 197	            & 6.722964            \\
4               & 8.6669e+00	                  & 5.4772e+00	                    & 3.1896e+00	                                & 7.8658e-16	                  & 2457	          & 80.679855           \\
5               & 1.3008e+01	                  & 7.4162e+00	                    & 5.5917e+00	                                & 9.9988e-16	                  & 64030	          & 2073.690887         \\
6               & 1.8819e+01	                  & 9.5394e+00	                    & 9.2799e+00	                                & 4.8088e-10	                  & 200000	        & -                   \\
7               & 2.7312e+01	                  & 1.1832e+01	                    & 1.5479e+01	                                & 3.2752e-11	                  & 200000	        & -                   \\
8               & 3.7854e+01	                  & 1.4283e+01	                    & 2.3571e+01	                                & 3.0592e-10	                  & 200000	        & -                   \\
9               & 7.0377e+01	                  & 1.6882e+01	                    & 5.3495e+01	                                & 1.3261e-09	                  & 200000	        & -
\end{tblr}
\caption{Результат работы алгоритма с диагональным предобуславливанием}
\end{table}

\newpage

\section{Неполное LU разложение}
\subsection{Теоретическая часть}

\begin{align}
  &L_{ii} = U_{ii} = \sqrt{A_{ii} - \sum_{k < i}L_{ik}U_{ki}} \\[1ex]
  &L_{ij} = \frac{1}{S_{jj}} \left[ A_{ij} - \sum_{k < j}L_{ik}U_{ki} \right] \\[1ex]
  &U_{ji} = \frac{1}{S_{jj}} \left[ A_{ji} - \sum_{k < j}L_{jk}U_{ki} \right]
\end{align}

\subsection{Исходный код}

\inputminted[firstline=104, lastline=170]{c}{/home/mehandes/c/src/github.com/meha4j/math/vec/mtx/src/mtx_csj.c}

\end{document}

