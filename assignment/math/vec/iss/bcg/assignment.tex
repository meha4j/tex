\documentclass[12pt, a4paper]{article}

\usepackage[
  margin=1in
]{geometry}

\usepackage[T1, T2A]{fontenc}
\usepackage[utf8]{inputenc}
\usepackage[russian]{babel}
\usepackage{minted}
\usepackage{hyperref}
\usepackage{tabularray}
\usepackage{amsmath}
\usepackage{float}

\begin{document}

\begin{titlepage}
  \centering
  \textsc{Новосибирский государственный технический университет}\par
  \vspace{1mm}
  Кафедра прикладной математики\par
  \vspace{4cm}
  \textsc{Практическая работа \textnumero 3}\par
  {\huge\bfseries Стабилизированный метод бисопряжённых градиентов\par}
  \vspace{1cm}
  {\scriptsize ФПМИ, ПМ-24\par}
  \vspace{1mm}
  {\itshape\large Параскун И., Герасименко В.\par}
  \vfill
  {\small преподаватели\par}
  \vspace{2mm}
  \textsc{Рояк М. Э., д.т.н., профессор}\par
  \vspace{1mm}
  \textsc{Задорожный А. Г., к.т.н., доцент}\par
  \vspace{1mm}
  \textsc{Леонович Д. А., ассистент}\par
  \vfill
  \large{Новосибирск, 2024}
\end{titlepage}

\newpage

\setcounter{page}{2}
\tableofcontents

\newpage

\section{Постановка задачи}
Реализовать стабилизированный метод бисопряжённых градиентов для матриц в строчно-столбцовом 
формате с учётом следующих требований:

\begin{itemize}
  \item Матрица должна обрабатываться в соответствии с форматом;
  \item Выход из итерационного процесса выполнять, если относительная 
    невязка стала меньше заданного параметра;
  \item Предусмотреть аварийный выход из итерационного процесса при достижении 
    максимального количества итераций;
\end{itemize}

\section{Исходный код}
\subsection{Стабилизированный метод бисопряжённых градиентов}
\subsection{Стабилизированный метод бисопряжённых градиентов c предобуславливанием}
\subsection{ILU разложение}

\section{Исследование}
\subsubsection{Матрица со значительным диагональным преобладанием}

\begin{table}[H]
\centering
\begin{tblr}{
  width=\textwidth, 
  colspec={|X|X|X|X|c|X|},
  rowspec={|c|c|}
}
\SetCell{c} $||\textbf{x}||$  & \SetCell{c} $||\textbf{x}^*||$ & \SetCell{c} $||\textbf{x}^* - \textbf{x}||$ & \SetCell{c} $||\textbf{r}||$ & \SetCell{c} $N$ & \SetCell{c} $t, ms$ \\
1.9621E+01	         & 1.9621E+01	           & 5.3658E-08	               & 9.9989E-11	         & 137040	         & 2.7349E+01
\end{tblr}
\caption{Результат работы алгоритма без предобуславливания}
\end{table}

\begin{table}[H]
\centering
\begin{tblr}{
  width=\textwidth, 
  colspec={|X|X|X|X|c|X|},
  rowspec={|c|c|}
}
\SetCell{c} $||\textbf{x}||$  & \SetCell{c} $||\textbf{x}^*||$ & \SetCell{c} $||\textbf{x}^* - \textbf{x}||$ & \SetCell{c} $||\textbf{r}||$ & \SetCell{c} $N$ & \SetCell{c} $t, ms$ \\
1.9621E+01	         & 1.9621E+01	           & 5.3658E-08	               & 9.9989E-11	         & 137040	         & 2.7349E+01
\end{tblr}
\caption{Результат работы алгоритма с ILU предобуславливанием}
\end{table}

\begin{table}[H]
\centering
\begin{tblr}{
  width=\textwidth, 
  colspec={|X|X|X|X|c|X|},
  rowspec={|c|c|}
}
\SetCell{c} $||\textbf{x}||$  & \SetCell{c} $||\textbf{x}^*||$ & \SetCell{c} $||\textbf{x}^* - \textbf{x}||$ & \SetCell{c} $||\textbf{r}||$ & \SetCell{c} $N$ & \SetCell{c} $t, ms$ \\
1.9621E+01	         & 1.9621E+01	           & 5.3658E-08	               & 9.9989E-11	         & 137040	         & 2.7349E+01
\end{tblr}
\caption{Результат работы алгоритма с диагональным предобуславливанием}
\end{table}


\subsubsection{Матрицы Гильберта}

\begin{table}[H]
\centering
\begin{tblr}{
  width=\textwidth, 
  colspec={|c|X|X|X|X|c|X|},
  rowspec={|c|ccccccc|}
}
\SetCell{c} $k$ & \SetCell{c} $||\textbf{x}||$  & \SetCell{c} $||\textbf{x}^*||$  & \SetCell{c} $||\textbf{x}^* - \textbf{x}||$ & \SetCell{c} $||\textbf{r}||$  & \SetCell{c} $N$ & \SetCell{c} $t, ms$ \\
3               & 1.9621E+01	                  & 1.9621E+01	                    & 5.3658E-08	                                & 9.9989E-11	                  & 137040	        & 2.7349E+01          \\
4               & 1.9621E+01	                  & 1.9621E+01	                    & 5.3658E-08	                                & 9.9989E-11	                  & 137040	        & 2.7349E+01          \\
5               & 1.9621E+01	                  & 1.9621E+01	                    & 5.3658E-08	                                & 9.9989E-11	                  & 137040	        & 2.7349E+01          \\
6               & 1.9621E+01	                  & 1.9621E+01	                    & 5.3658E-08	                                & 9.9989E-11	                  & 137040	        & 2.7349E+01          \\
7               & 1.9621E+01	                  & 1.9621E+01	                    & 5.3658E-08	                                & 9.9989E-11	                  & 137040	        & 2.7349E+01          \\
8               & 1.9621E+01	                  & 1.9621E+01	                    & 5.3658E-08	                                & 9.9989E-11	                  & 137040	        & 2.7349E+01          \\
9               & 1.9621E+01	                  & 1.9621E+01	                    & 5.3658E-08	                                & 9.9989E-11	                  & 137040	        & 2.7349E+01
\end{tblr}
\caption{Результат работы алгоритма без предобуславливания}
\end{table}

\begin{table}[H]
\centering
\begin{tblr}{
  width=\textwidth, 
  colspec={|c|X|X|X|X|c|X|},
  rowspec={|c|ccccccc|}
}
\SetCell{c} $k$ & \SetCell{c} $||\textbf{x}||$  & \SetCell{c} $||\textbf{x}^*||$  & \SetCell{c} $||\textbf{x}^* - \textbf{x}||$ & \SetCell{c} $||\textbf{r}||$  & \SetCell{c} $N$ & \SetCell{c} $t, ms$ \\
3               & 1.9621E+01	                  & 1.9621E+01	                    & 5.3658E-08	                                & 9.9989E-11	                  & 137040	        & 2.7349E+01          \\
4               & 1.9621E+01	                  & 1.9621E+01	                    & 5.3658E-08	                                & 9.9989E-11	                  & 137040	        & 2.7349E+01          \\
5               & 1.9621E+01	                  & 1.9621E+01	                    & 5.3658E-08	                                & 9.9989E-11	                  & 137040	        & 2.7349E+01          \\
6               & 1.9621E+01	                  & 1.9621E+01	                    & 5.3658E-08	                                & 9.9989E-11	                  & 137040	        & 2.7349E+01          \\
7               & 1.9621E+01	                  & 1.9621E+01	                    & 5.3658E-08	                                & 9.9989E-11	                  & 137040	        & 2.7349E+01          \\
8               & 1.9621E+01	                  & 1.9621E+01	                    & 5.3658E-08	                                & 9.9989E-11	                  & 137040	        & 2.7349E+01          \\
9               & 1.9621E+01	                  & 1.9621E+01	                    & 5.3658E-08	                                & 9.9989E-11	                  & 137040	        & 2.7349E+01
\end{tblr}
\caption{Результат работы алгоритма с ILU предобуславливанием}
\end{table}

\begin{table}[H]
\centering
\begin{tblr}{
  width=\textwidth, 
  colspec={|c|X|X|X|X|c|X|},
  rowspec={|c|ccccccc|}
}
\SetCell{c} $k$ & \SetCell{c} $||\textbf{x}||$  & \SetCell{c} $||\textbf{x}^*||$  & \SetCell{c} $||\textbf{x}^* - \textbf{x}||$ & \SetCell{c} $||\textbf{r}||$  & \SetCell{c} $N$ & \SetCell{c} $t, ms$ \\
3               & 1.9621E+01	                  & 1.9621E+01	                    & 5.3658E-08	                                & 9.9989E-11	                  & 137040	        & 2.7349E+01          \\
4               & 1.9621E+01	                  & 1.9621E+01	                    & 5.3658E-08	                                & 9.9989E-11	                  & 137040	        & 2.7349E+01          \\
5               & 1.9621E+01	                  & 1.9621E+01	                    & 5.3658E-08	                                & 9.9989E-11	                  & 137040	        & 2.7349E+01          \\
6               & 1.9621E+01	                  & 1.9621E+01	                    & 5.3658E-08	                                & 9.9989E-11	                  & 137040	        & 2.7349E+01          \\
7               & 1.9621E+01	                  & 1.9621E+01	                    & 5.3658E-08	                                & 9.9989E-11	                  & 137040	        & 2.7349E+01          \\
8               & 1.9621E+01	                  & 1.9621E+01	                    & 5.3658E-08	                                & 9.9989E-11	                  & 137040	        & 2.7349E+01          \\
9               & 1.9621E+01	                  & 1.9621E+01	                    & 5.3658E-08	                                & 9.9989E-11	                  & 137040	        & 2.7349E+01
\end{tblr}
\caption{Результат работы алгоритма с диагональным предобуславливанием}
\end{table}

\newpage

\subsection{Тестовые данные}
\subsubsection{Матрица с отрицательными внедиагональными элементами}
\begin{minted}{console}
  6  -3   0   0  -1   0   0  -1   0   0
 -2   5  -1   0   0  -2   0   0   0   0
  0  -1   6  -2   0   0  -2   0   0  -1
  0   0  -1   1   0   0   0   0   0   0
 -3   0   0  -3  13  -3   0   0  -4   0
  0  -2   0   0  -2   7   0   0   0  -3
  0   0   0   0   0  -4   5  -1   0   0
 -3   0   0  -4   0   0  -2  11  -2   0
  0  -4   0   0  -2   0   0   0  10  -4 
  0   0  -2   0   0  -2   0   0  -2   6 
\end{minted}

\noindent$\nu_A = ||A||*||A^{-1}|| = 1018.5$ \textit{(первая норма)} \\
\noindent$\nu_A = ||A||*||A^{-1}|| = 370.89$ \textit{(вторая норма)}

\subsubsection{Матрица с положительными внедиагональными элементами}
\begin{minted}{console}
  6   3   0   0   1   0   0   1   0   0
  2   5   1   0   0   2   0   0   0   0
  0   1   6   2   0   0   2   0   0   1
  0   0   1   1   0   0   0   0   0   0
  3   0   0   3  13   3   0   0   4   0
  0   2   0   0   2   7   0   0   0   3
  0   0   0   0   0   4   5   1   0   0
  3   0   0   4   0   0   2  11   2   0
  0   4   0   0   2   0   0   0  10   4 
  0   0   2   0   0   2   0   0   2   6 
\end{minted}

\noindent$\nu_A = ||A||*||A^{-1}|| = 71.980$ \textit{(первая норма)} \\
\noindent$\nu_A = ||A||*||A^{-1}|| = 42.194$ \textit{(вторая норма)}

\end{document}

